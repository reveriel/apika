\subsection{dex2oat}
/system/bin/dex2oat对应的源码文件位于/art/dex2oat/dex2oat.cc。main函数直接
调用art命名空间下的dex2oat函数,dex2oat函数的主要结构如下所示:
\begin{lstlisting}
//command-line argument parsing
......

//create oat file but not write data
std::unique_ptr<File> oat_file;
......

//create Dex2Oat
Dex2Oat::Create(...);
......

//extract classes.dex file
std::vector<const DexFile*> dex_files;
......

//create oat format file
dex2oat->CreateOatFile(...);
......

//create image file
dex2oat->CreateImageFile(...);
......
\end{lstlisting}
dex2oat函数代码比较长,前面一部分都是在解析
参数,对一些不正确的参数组合会调用Usage函数打印dex2oat的使用方法并退出。
在完成参数的解析判断之后会创建一个指向oat\_location的文件指针,代码如下:
\begin{lstlisting}
std::unique_ptr<File> oat_file;
bool create_file = !oat_unstripped.empty();
if (create_file) {
	oat_file.reset(OS::CreateEmptyFile(oat_unstripped.c_str()));
	if (oat_location.empty()) {
		oat_location = oat_filename;
	}
 } else {
	 oat_file.reset(new File(oat_fd, oat_location));
	 oat_file->DisableAutoClose();
	 oat_file->SetLength(0);
 }
\end{lstlisting}
声明文件指针变量oat\_file,if和else分支都是在创建一个File实例并赋值给oat\_file
,这部分代码仅仅是创建了一个File实例,没有真正写入oat格式的文件数据。

接着开始完成dex到oat的转换工作,声明指向Dex2Oat的指针p\_dex2oat,类Dex2Oat的定义
位于art/dex2oat/dex2oat.cc中,接着调用Dex2Oat的Create方法,
\begin{lstlisting}
Dex2Oat* p_dex2oat;
if (!Dex2Oat::Create(&p_dex2oat,
                       runtime_options,
                       *compiler_options,
                       compiler_kind,
                       instruction_set,
                       instruction_set_features,
                       verification_results.get(),
                       &method_inliner_map,
                       thread_count)) {
    LOG(ERROR) << "Failed to create dex2oat";
    return EXIT_FAILURE;
  }
}
\end{lstlisting}
Dex2Oat的Create方法代码如下:
\begin{lstlisting}
static bool Create(Dex2Oat** p_dex2oat,...){
	SHARED_TRYLOCK_FUNCTION(true, Locks::mutator_lock_) {
		CHECK(verification_results != nullptr);
		CHECK(method_inliner_map != nullptr);
		std::unique_ptr<Dex2Oat> dex2oat(new  Dex2Oat(&compiler_options,...))
		if (!dex2oat->CreateRuntime(runtime_options,instruction_set)) {
			*p_dex2oat = nullptr;
			return false;
		}
		*p_dex2oat = dex2oat.release();
		return true;
	}
}
\end{lstlisting}
Create方法调用CreateRuntime函数获取Runtime类的实例,Runtime使用单例模式,所以获取Runtime
实例前会先获取锁。接着创建Dex2Oat的实例,Dex2Oat的构造函数比较简单,只是一些
类成员的赋值,如下所示:
\begin{lstlisting}
explicit Dex2Oat(const CompilerOptions* compiler_options,...)
      : compiler_options_(compiler_options),
        compiler_kind_(compiler_kind),
        instruction_set_(instruction_set),
        instruction_set_features_(instruction_set_features),
        verification_results_(verification_results),
        method_inliner_map_(method_inliner_map),
        runtime_(nullptr),
        thread_count_(thread_count),
        start_ns_(NanoTime()) {
    CHECK(compiler_options != nullptr);
    CHECK(verification_results != nullptr);
    CHECK(method_inliner_map != nullptr);
  }
\end{lstlisting}
其中start\_ns\_记录实例化的时间。

回到dex2oat函数,接着将p\_dex2oat赋值给dex2oat变量,获取当前线程,将线程从
runnable切换到suspend状态,释放掉之前Dex2Oat创建时获得的锁。
WellKnownClasses::Init函数用来完成一些JNI类、函数、以及字段的初始化操作,函数
代码位于art/runtime/well\_known\_classes.cc.

接着声明vector变量dex\_files保存相应的DexFile实例,if分支的GetBootClassPath函数
以及else分支的DexFile::OpenFromZip函数都是获取相应的dex文件。

接着通过EnableWrite()将dex文件设置为可写,保证能够进行dex到dex的优化转换。

最后调用CreateOatFile和CreateImageFile函数来分别创建oat文件和镜像文件。这里我们
只关心CreateOatFile函数的实现过程。
\begin{lstlisting}
const CompilerDriver* CreateOatFile(...) {
    ......
    std::unique_ptr<CompilerDriver> driver(new CompilerDriver(...));
    driver->GetCompiler()->SetBitcodeFileName(*driver.get(), bitcode_filename);
    driver->CompileAll(class_loader, dex_files, &timings);

    OatWriter oat_writer(...);
    ......

    if (!driver->WriteElf(android_root, is_host, dex_files, &oat_writer, oat_file)) {
    ......
    }

    ......
}
\end{lstlisting}
CreateOatFile函数创建CompilerDriver实例,调用CompilerDriver类的CompileAll方法
(源码位于/art/compiler/driver/compiler\_driver.cc)来完成dex到IR的转换。
CompileAll方法的代码如下:
\begin{lstlisting}
void CompilerDriver::CompileAll(jobject class_loader,
                                const std::vector<const DexFile*>& dex_files,
                                TimingLogger* timings) {
  DCHECK(!Runtime::Current()->IsStarted());
  std::unique_ptr<ThreadPool> thread_pool(new ThreadPool("Compiler driver thread pool", thread_count_ - 1));
  PreCompile(class_loader, dex_files, thread_pool.get(), timings);
  Compile(class_loader, dex_files, thread_pool.get(), timings);
  if (dump_stats_) {
    stats_->Dump();
  }
}
\end{lstlisting}
CompileAll主要包含两个函数:PreCompile和Compile。
ProCompile函数的代码:
\begin{lstlisting}
void CompilerDriver::PreCompile(jobject class_loader,
             const std::vector<const DexFile*>& dex_files,
             ThreadPool* thread_pool, TimingLogger* timings) {
    LoadImageClasses(timings);
    Resolve(class_loader, dex_files, thread_pool, timings);
    if (!compiler_options_->IsVerificationEnabled()) {
    LOG(INFO) << "Verify none mode specified, skipping verification.";
     SetVerified(class_loader, dex_files, thread_pool, timings);
     return;
   }
 
  Verify(class_loader, dex_files, thread_pool, timings);
 
  InitializeClasses(class_loader, dex_files, thread_pool, timings);
 
   UpdateImageClasses(timings);
 }

\end{lstlisting}
PreCompile主要是进行编译前的一些预操作,主要是类的加载、解析、校验和初始化等。

Compile函数直接调用CompileDexFile函数,该函数的代码:
\begin{lstlisting}
void CompilerDriver::CompileDexFile(jobject class_loader, const DexFile& dex_file,
                                    const std::vector<const DexFile*>& dex_files,
                                    ThreadPool* thread_pool, TimingLogger* timings) {
  TimingLogger::ScopedTiming t("Compile Dex File", timings);
  ParallelCompilationManager context(Runtime::Current()->GetClassLinker(), class_loader, this,
                                     &dex_file, dex_files, thread_pool);
  context.ForAll(0, dex_file.NumClassDefs(), CompilerDriver::CompileClass, thread_count_);
}
\end{lstlisting}
通过类ParallelCompilationManager并行编译dex文件中的各个类。
CompileClass通过调用CompilerDriver类的CompileMethod来编译每个类对应的
direct methods和vitual methods。CompileMethod方法的代码如下:
\begin{lstlisting}
void CompilerDriver::CompileMethod(const DexFile::CodeItem* code_item, uint32_t access_flags,
                                   InvokeType invoke_type, uint16_t class_def_idx,
                                   uint32_t method_idx, jobject class_loader,
                                   const DexFile& dex_file,
                                   DexToDexCompilationLevel dex_to_dex_compilation_level) {
  CompiledMethod* compiled_method = nullptr;
  uint64_t start_ns = kTimeCompileMethod ? NanoTime() : 0;

  if ((access_flags & kAccNative) != 0) {
    if (!compiler_options_->IsCompilationEnabled() &&
        (instruction_set_ == kX86_64 || instruction_set_ == kArm64)) {
    } else {
      compiled_method = compiler_->JniCompile(access_flags, method_idx, dex_file);
      CHECK(compiled_method != nullptr);
    }
  } else if ((access_flags & kAccAbstract) != 0) {
  } else {
    MethodReference method_ref(&dex_file, method_idx);
    bool compile = verification_results_->IsCandidateForCompilation(method_ref, access_flags);
    if (compile) {     
      compiled_method = compiler_->Compile(code_item, access_flags, invoke_type, class_def_idx,
                                           method_idx, class_loader, dex_file);
    }
    if (compiled_method == nullptr && dex_to_dex_compilation_level != kDontDexToDexCompile) {
      (*dex_to_dex_compiler_)(*this, code_item, access_flags,
                              invoke_type, class_def_idx,
                              method_idx, class_loader, dex_file,
                              dex_to_dex_compilation_level);
    }
  }

  ......

  Thread* self = Thread::Current();
  if (compiled_method != nullptr) {
    MethodReference ref(&dex_file, method_idx);
    DCHECK(GetCompiledMethod(ref) == nullptr) << PrettyMethod(method_idx, dex_file);
    {
      MutexLock mu(self, compiled_methods_lock_);
      compiled_methods_.Put(ref, compiled_method);
    }
    DCHECK(GetCompiledMethod(ref) != nullptr) << PrettyMethod(method_idx, dex_file);
  }

  if (self->IsExceptionPending()) {
    ScopedObjectAccess soa(self);
    LOG(FATAL) << "Unexpected exception compiling: " << PrettyMethod(method_idx, dex_file) << "\n"
        << self->GetException(nullptr)->Dump();
  }
}
\end{lstlisting}
对于本地方法native method,如果编译选项IsCompilationEnabled为假并且指令集为
64位X86或者64位Arm时不做任何处理会触发通用的JNI调用,其他情况会调用编译器的
JniCompile方法,Android ART中存在两种JniCompile方法:一种位于
/art/compiler/compiler.cc,通过调用ArtLLVMJniCompileMethod
(位于/art/compiler/llvm/compiler\_llvm.cc)完成。ArtLLVMJniCompileMethod
是通过调用llvm JNI编译器的Compile方法(源码位于
/art/compiler/jni/portable/jni\_compiler.cc)来完成本地方法的编译。
另外一种位于/art/compiler/compilers.cc,通过调用ArtQuickCompileMethod
(位于/art/compiler/dex/frontend.cc)完成。ArtQuickCompileMethod最终是通过
调用CompileMethod方法完成编译工作。CompileMethod(/art/compiler/dex/frontend.cc)
主要是通过创建MIR Graph,对MIR Graph进行处理来完成方法的编译。

对于抽象方法,不做任何处理。

其他方法通过调用编译器的Compile方法进行处理。因为编译器选项有3个值:
Quick、Optimizing、Portable,所以对应的有3种Compile方法。
QuickCompiler::Compile(源码位于/art/compiler/compilers.cc)先调用
TryCompileWithSeaIR得到method,若method为空,则返回ArtQuickCompileMethod。
TryCompileWithSeaIR最终是调用CompileMethodWithSeaIr
(源码位于/art/compiler/sea\_ir/frontend.cc),
通过得到dex文件的SeaGraph,基于SeaGraph进行操作,完成方法的编译。

OptimizingCompiler::Compile(源码位于/art/compiler/compilers.cc)先调用
TryCompile,如果得到的method为空,则调用QuickCompiler::Compile完成编译。
TryCompile位于/art/compiler/optimizing/optimizing\_compiler.cc,TryCompile
的代码比较复杂,主要包含两个部分:
通过HGraphBuilder构建HGraph(方法对应的控制流图),基于HGraph进行一系列处理
包括构建支配树、转换为SSA形式、活跃分析等;
通过CodeGenerator创建代码生成器。

PortableCompiler::Compile(源码位于/art/compiler/compiler.cc)先调用           
TryCompileWithSeaIR得到method,若method为空,则返回ArtCompileMethod。
ArtCompileMethod调用CompileMethod方法完成方法的编译。
CompileMethod的处理流程:

1、创建CompilationUnit对象来存放一次编译中需要的信息,compilationunit相应成员的初始化; 

2、将dex文件中的Dalvik字节码解码为DecodedInstruction(这个结构体的定义位于
/art/compiler/dex/mir\_graph.h),并创建对应的MIR节点;

3、定位基本块的边界,并创建相应的BasicBlock对象,将MIR塞进去; 

4、确定控制流关系,将基本块连接起来构成控制流图(CFG),并添加恢复解释器状态和异常处理用的基本块;
 
5、将基本块都加到CompilationUnit里去;

6、MIR优化(带有局部优化和全局优化)
