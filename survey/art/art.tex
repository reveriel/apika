\section{Android ART analysis}
Android源码目前更新到5.1.1\_r2,\url{http://socialcompare.com/en/comparison/android-versions-comparison}
上给出了Android历史版本的比较,这里截取了5.0版本之后的比较,如表\ref{androidcompare}所示:
\begin{table}[H]
\caption{\label{androidcompare}\hei{android version comparison}}
\begin{tabular}{|p{2cm}|p{5cm}|p{3.5cm}|p{2.5cm}|}
\hline
version & key user features added & key developer features added & release date\\
\hline
android5.0 & New design (Material design); 
Speed improvement; 
Battery consumption improvement 
& Several new API; 
Tracking battery consumption app & 2014 Oct 17\\
\hline
android5.0.1 & bug fixes, fix issues with video playback and password failures & & 2014 Dec 2\\
\hline
android5.0.2 & Performance improvements and bug fixes & & 2014 Dec 19\\
\hline
android5.1 & Multiple SIM cards support;
Quick settings shortcuts to join Wi-Fi networks or control Bluetooth devices;
Lock protection if lost or stolen;
High Definition voice call
Stability and performance enhancements & & 2015 Mar 9\\
\hline
android5.1.1 & Speed improvement;
Bug fixes & & 2015 April 21\\
\hline
\end{tabular}
\end{table}

本文档分析的源码版本是android5.0.0\_r7,简单看了一下5.0到5.1.1版本的更新日志
\url{http://changes.droidsec.org/},关于ART的修改是在5.0.0到5.0.1;
5.0.2到5.1.0,主要是一些细节的修改。本文档主要是分析ART运行过程的大体框架,没有涉及到
一些细节的深入分析,所以与最新版本应该没有太大出入(\redt{后面再重新结合最新的源码修改本文档})。

\subsection{Android Runtime}
Android系统init进程启动过程中会运行app\_process进程,app\_process对应的源码
位于/frameworks/base/cmds/app\_main.cpp,在main函数中会启动Zygote:
\begin{lstlisting}
if (zygote) {
	        runtime.start("com.android.internal.os.ZygoteInit", args);
}
\end{lstlisting}
runtime是AppRuntime的实例,AppRuntime继承自AndroidRuntime.进入AndroidRuntime
类的start函数(位于/frameworks/base/core/jni/AndroidRuntime.cpp),部分代码
如下:
\begin{lstlisting}
void AndroidRuntime::start(const char* className, const Vector<String8>& options)
{
	......

	JniInvocation jni_invocation;					  
  jni_invocation.Init(NULL);
	JNIEnv* env;
    if (startVm(&mJavaVM, &env) != 0) {
     		return;
	}
       
	......
}
\end{lstlisting}	
调用JniInvocation类的Init函数(位于/libnativehelper/JniInvocation.cpp),
部分代码如下:
\begin{lstlisting}
bool JniInvocation::Init(const char* library) {
  library = GetLibrary(library);

  handle_ = dlopen(library, RTLD_NOW);
  
  ......

  if (!FindSymbol(reinterpret_cast<void**>(&JNI_GetDefaultJavaVMInitArgs_),
                  "JNI_GetDefaultJavaVMInitArgs")) {
    return false;
  }
  if (!FindSymbol(reinterpret_cast<void**>(&JNI_CreateJavaVM_),
                  "JNI_CreateJavaVM")) {
    return false;
  }
  if (!FindSymbol(reinterpret_cast<void**>(&JNI_GetCreatedJavaVMs_),
                  "JNI_GetCreatedJavaVMs")) {
    return false;
  }
  return true;
}
\end{lstlisting}
Init通过GetLibrary函数从属性系统获得名为persist.sys.dalvik.vm.lib的属性值,
dalvik模式,其值为libdvm.so,art模式下,其值为libart.so.通过dlopen打开对应的
库文件,这里我们只考虑libart.so.然后调用FindSymbol函数从打开的libart.so文件
中搜索到对应的导出符号,比如JNI\_CreateJavaVM对应的是
/art/runtime/jni\_internal.cc中的JNI\_CreateJavaVM函数。
回到AndroidRuntime::start函数,接着会调用startVM函数启动虚拟机,该函数最终
会调用JNI\_CreateJavaVM函数,ART模式对应的部分代码如下:
\begin{lstlisting}
extern "C" jint JNI_CreateJavaVM(JavaVM** p_vm, JNIEnv** p_env, void* vm_args) {
	const JavaVMInitArgs* args = static_cast<JavaVMInitArgs*>(vm_args);
	......
	RuntimeOptions options;
    for (int i = 0; i < args->nOptions; ++i) {
		JavaVMOption* option = &args->options[i];
	   	options.push_back(std::make_pair(std::string(option->optionString),
					    	option->extraInfo));
  	}
	if (!Runtime::Create(options, ignore_unrecognized)) {
		......
	}
	Runtime* runtime = Runtime::Current();
	bool started = runtime->Start();
	......
}
\end{lstlisting}
JNI\_CreateJavaVM函数首先解析虚拟机启动参数存入到Runtime::Options实例中,
根据解析的参数信息,调用Create函数创建Runtime实例,获取Runtime的当前实例,
接着调用Runtime的Start函数(位于art/runtime/runtime.cc),该函数代码如下:
\begin{lstlisting}
bool Runtime::Start() {
	......
	Thread* self = Thread::Current();
    self->TransitionFromRunnableToSuspended(kNative);
    ......
	InitNativeMethods();
	......
	if (is_zygote_) {
		if (!InitZygote()) {
		   	return false;
		}
	 } else {
		 DidForkFromZygote(...);
     }
	StartDaemonThreads();
    ......
}
\end{lstlisting}
Start函数获取当前运行线程,将该线程状态从Runnable切换到Suspend,通过
InitNativeMethods完成Native函数的注册。调用InitZygote完成一些文件系统的
增加。最后调用StartDaemonThreads启动守护进程。




