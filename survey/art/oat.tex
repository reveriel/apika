\subsection{oat执行}
ART运行时提供了一个OatFile类,通过调用它的静态成员函数Open可以在本进程中加载OAT文件,这个函数定义在文件art/runtime/oat\_file.cc中,
\begin{lstlisting}
OatFile* OatFile::Open(const std::string& filename,
                       const std::string& location,
                       byte* requested_base,
                       bool executable,
                       std::string* error_msg) {
  .............

}
\end{lstlisting}
参数filename和location指向要加载的OAT文件,参数requested\_base是一个可选参数,用来描述要加载的OAT文件里面的oatdata段要加载的位置,参数executable表示要加载的OAT是不是应用程序的主执行文件。一般来说,一个应用程序只有一个classes.dex文件, 这个classes.dex文件经过编译后,就得到一个OAT主执行文件。不过,应用程序也可以在运行时动态加载DEX文件。这些动态加载的DEX文件在加载的时候同样会被翻译成OAT再运行,就不属于主执行文件了。

ART运行时利用LLVM编译框架将DEX字节码翻译成本地机器指令,这些生成的机器指令就保存在ELF文件格式的OAT文件的oatexec段中.ART运行时会为每一个类方法都生成一系列的本地机器指令,这些本地机器指令不是孤立存在的,因为它们可能需要其它的函数来完成自己的功能。这要求Backend为类方法生成本地机器指令时,要处理调用其它模块提供的函数的问题。ART运行时支持两种类型的Backend:Portable和Quick。

Portable类型的Backend通过集成在LLVM编译框架里面的一个称为MCLinker的链接器来生成本地机器指令。这些OAT文件要通过系统的动态链接器提供的dlopen函数来加载。函数dlopen在加载OAT文件的时候,会通过重定位技术来处理好它与其它模块的依赖关系,使得它能够调用其它模块提供的接口。

Quick类型的Backend生成的本地机器指令用另外一种方式来处理依赖模块之间的依赖关系,ART运行时会在每一个线程的TLS(线程本地区域)提供一个函数表,Quick类型的Backend生成的本地机器指令通过引用这个函数表来调用其它模块的函数。也就是说,Quick类型的Backend生成的本地机器指令要依赖于ART运运时提供的函数表,这使得Quick类型的Backend生成的OAT文件在加载时不需要重定位,因此就不需要通过系统的动态链接器提供的dlopen函数来加载。由于省去重定位这个操作,Quick类型的Backend生成的OAT文件在加载时就会更快,这也是称为Quick的缘由。

如果在编译ART运行时时,定义了宏ART\_USE\_PORTABLE\_COMPILER,那么就表示要使用Portable类型的Backend来生成OAT文件,否则就使用Quick类型的Backend来生成OAT文件。Open函数的实现过程:

1. 如果编译时指定了ART\_USE\_PORTABLE\_COMPILER宏,并且参数executable为true,那么就通过OatFile类的静态成员函数OpenDlopen来加载指定的OAT文件。OatFile类的静态成员函数OpenDlopen直接通过动态链接器提供的dlopen函数来加载OAT文件。

2. 其余情况下,通过OatFile类的静态成员函数OpenElfFile来加载指定的OAT文件。这种方式是按照ELF文件格式来解析要加载的OAT文件的,并且根据解析获得的信息将OAT里面相应的段加载到内存中来。

OatFile类的成员函数Dlopen首先是通过动态链接器提供的dlopen函数将参数elf\_filename指定的OAT文件加载到内存中来,接着同样是通过动态链接器提供的dlsym函数从加载进来的OAT文件获得两个导出符号oatdata和oatlastword的地址,最后调用成员函数Setup来解析已经加载内存中的oatdata段,以获得ART运行所需要的更多信息。

函数Setup定义在文件art/runtime/oat\_file.cc中,通过分析Setup函数更好地了解oat文件格式。

函数Setup的一开始调用了函数GetOatHeader,因此OAT文件里面的oatdata段的开始储存着一个OAT头,这个OAT头通过类OatHeader描述,定义在文件art/runtime/oat.h中

然后调用函数GetImageFileLocationSize得到正在打开的OAT依赖的Image空间文件的路径大小,因此紧接着在OAT头后面的是Image空间文件路径

接着的代码获得包含在oatdata段的DEX文件描述信息,每一个DEX文件记录在oatdata段的描述信息包括:

DEX文件路径大小,保存在变量dex\_file\_location\_size中

DEX文件路径,保存在变量dex\_file\_location\_data中

DEX文件检验和,保存在变量dex\_file\_checksum中

DEX文件内容在oatdata段的偏移,保存在变量dex\_file\_offset中

DEX文件包含的类的本地机器指令信息偏移数组,保存在变量methods\_offsets\_pointer中

上述得到的每一个DEX文件的信息都被封装在一个OatDexFile对象中,以便以后可以直接访问,在OAT文件中,每一个DEX文件包含的每一个类的描述信息都通过一个OatClass对象来描述
\begin{lstlisting}
OatFile::OatClass::OatClass(const OatFile* oat_file,
                            mirror::Class::Status status,
                            OatClassType type,
                            uint32_t bitmap_size,
                            const uint32_t* bitmap_pointer,
                            const OatMethodOffsets* methods_pointer)
    : oat_file_(oat_file), status_(status), type_(type),
      bitmap_(bitmap_pointer), methods_pointer_(methods_pointer) {}
\end{lstlisting}
参数oat\_file描述的是宿主OAT文件,参数status描述的是OAT类状态,参数methods\_pointer是一个数组,描述的是OAT类的各个方法的信息,它们被分别保存在OatClass类的相应成员变量中。通过这些信息,我们就可以获得包含在该DEX文件里面的类的所有方法的本地机器指令信息。
