\section{APK结构}
APK文件:Android application package文件。每个要安装到Android平台的应用都要被编译打包为一个单独的文件,后缀名为.apk,其中包含了应用的二进制代码、资源、配置文件等。

apk文件实际是一个zip压缩包,可以通过解压缩工具解开。
一般解压后的结构如表~\ref{tab:apk}所示:
\begin{table}[htbp]
\centering
\caption{\label{tab:apk}\hei{apk组成}}
\begin{tabular}{|c|c|}
\hline
文件或目录 & 作用\\
\hline
assets/ & 存放资源文件的目录\\
\hline
META-INF/ & 从java jar文件引入的描述包信息的目录\\
\hline
res/ & 存放资源文件的目录\\
\hline
libs/ & 如果存在的话,存放的是ndk编出来的so库\\
\hline
AndroidManifest.xml & 程序全局配置文件\\
\hline
classes.dex & 最终生成的dalvik字节码\\
\hline
resources.ars & 编译后的二进制资源文件\\
\hline
\end{tabular}
\end{table}

\begin{itemize}
\item assets目录\\
assets目录可以存放一些配置文件,这些文件的内容在程序运行过程中可以通过相关的API获得。

\item META-INF目录\\
META-INF目录下存放的是签名信息,用来保证apk包的完整性和系统的安全。在eclipse编译生成一个apk包时,会对所有要打包的文件做一个校验计算,并把计算结果放在META-INF目录下,这就保证了apk包里的文件不能被随意替换。

\item res目录\\
res目录存放资源文件,包括图片,字符串等等。

\item libs目录\\
lib目录下的子目录armeabi存放的是一些so文件。eclipse在打包的时候会根据文件名的命名规则(lib****.so)去打包so文件,开头和结尾必须分别为"lib"和".so",否则是不会打包到apk文件中的。

\item AndroidManifest.xml\\
该文件是每个应用都必须定义和包含的,它描述了应用的名字、版本、权限、引用的库文件等等信息,如要把apk上传到Google Market上,也要对这个xml做一些配置。在apk中的AndroidManifest.xml是经过压缩的,可以通过AXMLPrinter2工具解开,具体命令为:java -jar AXMLPrinter2.jar AndroidManifest.xml。

\item classes.dex文件\\
 classes.dex是java源码编译后生成的java字节码文件。但由于Android使用的dalvik虚拟机与标准的java虚拟机是不兼容的,dex文件与class文件相比,不论是文件结构还是opcode都不一样。目前常见的java反编译工具都不能处理dex文件。Android模拟器中提供了一个dex文件的反编译工具,dexdump。

\item resources.ars文件\\
编译后的二进制资源文件。
\end{itemize}

\redt{补充说明:}\\
res和assets文件夹来存放不需要系统编译成二进制的文件,例如字体文件等,两者的区别:

1.assets:不会在R.java文件下生成相应的标记,assets文件夹可以自己创建文件夹,必须使用AssetsManager类进行访问,存放到这里的资源在运行打包的时候都会打入程序安装包中。

2.res:会在R.java文件下生成标记,这里的资源会在运行打包操作的时候判断哪些被使用到了,没有被使用到的文件资源是不会打包到安装包中的。

