\section{Android游戏软件案例分析概述}
本文档中对于Android游戏软件的分析主要是基于以下两个方向进行,
一个是从dex文件本身入手,利用dex文件解析结果进行案例分析,
参见第\ref{dexanalysis}节;
另外一个是从游戏软件包含的动态链接库入手,分析动态链接库包含的信息,参见第\ref{soanalysis}节。

我们对游戏软件分析的\hei{主要过程}是:
首先对需要进行分析的游戏软件进行信息收集,
包括游戏软件apk文件的大小,其中包含的dex文件的大小、动态链接库及其大小,
参见第\ref{sec:dexanalysis:case}节;
然后基于DexAnalyzer对游戏软件的dex文件进行解析,
统计游戏软件中自定义方法、库方法和自定义的native方法
以及这些方法所属的类信息,并作相应的总结,
参见第\ref{sec:dexanalysis:casestudy}节。

dex文件解析是使用我们自己编写的DexAnalyzer工具对dex文件进行解析。
DexAnalyzer能统计游戏软件的dex文件中调用的各类方法及其从属的类信息,
各类方法主要分为游戏软件自身定义的方法和使用的库方法,
对于自定义方法进一步识别出其中的native method。

动态链接库分析是利用objdump工具对游戏软件的动态链接库文件(.so文件)
所包含的函数等进行深入分析。
我们将其中的函数分成五类:
游戏引擎库函数、
Lua运行时函数、
系统级JNI本地函数、
C++与Lua之间的绑定函数、
应用级JNI本地函数,
参见第\ref{sec:so:cocolib}节到第\ref{sec:so:appjnilib}节;
然后就每一类函数进行分析总结。

本文档中以MoonWarriors游戏软件为例,
对其classes.dex文件和动态链接库libgame.so进行分析。
依据dex文件解析数据,
我们发现有关lua的函数没有出现在dex字节码中;
随后从动态链接库文件中查看,
这些lua函数的实现均包含在动态链接库中,
参见第\ref{sec:so:lualib}节和第\ref{sec:so:bindlib}节。
而对于native方法,在dex字节中出现但没有具体的实现,
其实现也是包含在动态链接库中,参见\ref{sec:so:appjnilib}。

通过第\ref{dexanalysis}节和第\ref{soanalysis}节
对MoonWarriors游戏软件的分析使得
我们对Android游戏软件中调用的类和方法的信息有了初步的了解,
在第\ref{sec:dexanalysis:conclusion}节和第\ref{sec:so:conclusion}节
给出初步的总结。
%基于课题组的研究,后续还需要进行更深入的分析总结。


