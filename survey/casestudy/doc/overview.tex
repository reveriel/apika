\section{Android游戏软件案例分析概述}
本文档中对于Android游戏软件的分析主要是基于以下两个方向进行,一个是从dex文件本身入手,利用dex文件解析结果进行案例分析,参见第\ref{dexanalysis}节;另外一个是从游戏软件包含的动态链接库入手,分析动态链接库包含的信息,参见第\ref{soanalysis}节。

dex文件解析基于DexAnalysis[\ref{dextools}]工具对dex文件进行解析,将游戏软件中调用的方法进行分类,主要分为自定义方法和库方法,再进一步分析自定义方法中的native method。分析过程首先对需要进行分析的游戏软件进行信息收集,包括游戏软件apk大小、dex文件大小、动态链接库及其大小,参见第\ref{sec:dexanalysis:case}节,然后基于DexAnalysis对游戏软件的dex文件进行解析,统计游戏软件中自定义方法、库方法和native方法以及这些方法对应的类信息,并作相应的总结,参见第\ref{sec:dexanalysis:casestudy}节。

动态链接库分析是利用objdump工具对游戏软件的动态链接库包含的函数等进行深入分析。将其中的函数分成五类:游戏引擎库函数、Lua运行时函数、系统级JNI本地函数、C++与Lua之间的绑定函数、应用级JNI本地函数,参见第\ref{sec:so:cocolib}节到第\ref{sec:so:appjnilib}节,然后就每一类函数进行分析总结。

本文档中以MoonWarriors游戏软件为例,对其classes.dex文件和动态链接库libgame.so进行分析。依据dex文件解析数据,我们发现有关lua的函数没有出现在dex字节码中,随后从动态链接库中查看,这些lua函数的实现均包含在动态链接库中,参见第\ref{sec:so:lualib}节和第\ref{sec:so:bindlib}节。而对于native方法,在dex字节中出现但没有具体的实现,其实现也是包含在动态链接库中,参见\ref{sec:so:appjnilib}。

第\ref{dexanalysis}节和第\ref{soanalysis}节对MoonWarriors游戏软件的分析使得我们对Android游戏软件中调用的类和方法的信息有了初步的了解,在第\ref{sec:dexanalysis:conclusion}节和第\ref{sec:so:conclusion}节给出初步的总结。基于课题组的研究,后续还需要进行更深入的分析总结。


