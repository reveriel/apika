\subsection{dex文件结构}
\begin{table}[H]
\caption{\label{fig:dex}\hei{dex文件结构}}
\begin{tabular}{|p{1.6cm}|p{2.7cm}|p{8.5cm}|}
\hline
名称 & 格式 & 描述\\
\hline
header & header\_item & 文件头\\
\hline
string\_ids & string\_id\_item[] & 字符串标识符列表。这里标志了当前DEX文件所有字符串使用的标志,也有一些内部名称(例如.,类型描述)或者代码段引用的一些常量对象。这个表按照字符串常量进行排序,字符串使用UTF-16进行编码(不在本地敏感方式),并且这个表不包含任何复制项。\\
\hline
type\_ids & type\_id\_item[] & 类型标识项。这里是此文件中所有类型的标识(类,队列,或者主类型),无论在文件中是否定义。这个表按照字符串标识索引进行排序,它不包含任何复制项。\\
\hline
proto\_ids & proto\_id\_item[] & 函数原型标识表。这里定义了此文件中所用引用的原型标识。这个表按照类型标识索引进行排序,并且参数也是通过类型标识索引进行排序。此表不包含复制项。\\
\hline
field\_ids & field\_id\_item[] & 区域标识符列表。这里定义了在此文件中的所有区域定义,此表按照类型标识索引进行主排序,并且按照区域名称作为中排序,按照类型作为子排序。并不包含复制项。\\
\hline
method\_ids & method\_id\_item[] & 函数标识表。定义了此文件引用的方法标识。按照类型标识索引作为主排序,按照方法名作为辅排序,按照方法原型作为子排序。不包含复制项。\\
\hline
class\_defs & class\_def\_item[] & 类标识表。引用这些类的父类以及接口前必须要经过排序。并且此列表中不能出现重复的类名。\\
\hline
data & ubyte[] & 以上表列出的数据,在此进行保存。\\
\hline
link\_data & ubyte[] & 静态链接数据段。如果此节为空则没有静态链接文件。\\
\hline
\end{tabular}
\end{table}

dex文件格式如表~\ref{fig:dex}所示,具体参考dalvik虚拟机源代码介绍官网\url{https://source.android.com/devices/tech/dalvik/dex-format.html}
