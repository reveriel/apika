\section{基于cocos2d-x的游戏软件案例分析:so文件解析}
\label{soanalysis}
我们以MoonWarriors软件包含的动态链接库libgame.so为例,
对该动态链接库包含的函数等进行深入分析。
我们用objdump命令,从libgame.so中导出符号表。
符号表中的信息可以参见
\url{https://ring0.me/2014/12/linkers-loaders-library-notes/}文章最后的表格。
我们将其中的函数分成如下五类,同时结合cocos2d-x 3.3版本的源码分析这几类函数在源码中的位置,
依次在下面各小节中分别说明:
\begin{enumerate}
\item \hei{游戏引擎库函数}:
	指由cocos2d-x引擎提供的游戏相关的库所包含的函数,
	参见第\ref{sec:so:cocolib}节;
\item \hei{Lua运行时函数}:
	Lua运行时环境(Lua虚拟机)提供的库函数,
	参见第\ref{sec:so:lualib}节;
\item \hei{系统级JNI本地函数}:
	cocos2d-x中针对android系统中的JNI函数给出的本地函数实现,
	参见第\ref{sec:so:sysjnilib}节;
\item \hei{C++与Lua之间的绑定函数}:
	cocos2d-x用于方便沟通C/C++层与Lua vm之间绑定(binding)所使用的、
	函数名以tolua开头的函数(一个自动bind工具),
	参见第\ref{sec:so:bindlib}节;
\item \hei{应用级JNI本地函数}:
	以Java\_org\_cocos2dx\_lib打头的函数,
	这些是针对cocos2d以及游戏应用的JNI函数给出的本地函数实现,
	参见第\ref{sec:so:appjnilib}节;
\end{enumerate}

表~\ref{table:so:fun}列出了libgame.so中涉及的5类函数的个数及所占比例。
\begin{table}[H]
\begin{center}
\caption{\hei{各类符号的数目及比例}}
\label{table:so:fun}
\begin{tabular}{|c|c|c|}
\hline 类别 & 该类符号的数目 & 100\% \\
\hline cocos2d-x游戏库函数& 8431 & 52.69\%  \\
\hline  lua vm 提供的API & 74 & 0.45\% \\
\hline C/C++与lua之间的绑定函数 & 79 &  0.4808\% \\
\hline 与JNI相关的函数(不含接口JNI) & 69 & 0.42\%\\
\hline 接口JNI & 21 & 0.128\%\\
\hline 总函数数目 & 16431 & 100\%  \\
\hline
\end{tabular}
\end{center}
\end{table}

\subsection{cocos2d-x游戏库函数}
\label{sec:so:cocolib}
这类函数是指由cocos2d-x引擎提供的游戏相关的库所包含的函数,
主要位于cocos2d-x-3.3/cocos/base目录下。

首先根据
objdump -T libgame.so|grep ".text" > \_text.txt \\
命令得到文件\_text.txt,
此文件中的每一行的最后一项均表示一个游戏在运行过程中可能用到的函数名。
然后我们查看该文件,发现共有16431行,
也就是说游戏运行过程中可能用到16431个函数。

而通过执行:

cat \_text.txt | grep  "\_Z" > \_z.txt

cat \_z.txt | grep "cocos2dx" > cocos2dx.txt\\
查看得到的cocos2dx.txt,发现共有8431行,
可以推出cocos2d-x游戏引擎提供了8431个库函数,
这些函数都是经过编译器去除面向对象特性后得到的函数。例如:
 
\begin{lstlisting}
  _ZTSN7cocos2d9extension9CCBReaderE
  _ZN7cocos2d10CCDirector13setProjectionENS_20ccDirectorProjectionE
  _ZTVN7cocos2d8CCActionE
  _ZTIN7cocos2d8CCLiquidE
  _ZNK7cocos2d13CCDictElement9getStrKeyEv
  _ZN7_JNIEnv16CallObjectMethodEP8_jobjectP10_jmethodIDz
\end{lstlisting}

这些函数的函数名特点如下:
	
以\_ZN7\_JNIEnv16CallObjectMethodEP8\_jobjectP10\_jmethodIDz 为例,它就是编译器将名字空间JNIEnv中
的CallObjectMethod(unsigned char,\_jobject* ,\_jmethodID* )函数去除面向对象特征转换后得到的
能唯一标识的符号名字。
\begin{enumerate}
\item \_Z 是所有改名后的符号的开始符号;
\item N表示其后跟的是namespace或者类名;
\item 7\_JNIEnv 表示长度为7个字符的类名或名字空间\_JNIEnv;
\item 16CallObectMethod表示由16个字符组成的函数名;
\item 再接下去就是函数参数类型列表,依次描述每个参数的类型:
\begin{itemize}
\item E 表示第1个参数为unsigned char类型;
\item P 表示指针类型,由于其后是8\_jobject,故第2个参数类型为\_jobject *;
\item 同理有第3个参数类型为\_jmethodID *;
\item z表示参数类型列表的结束。
\end{itemize}
\end{enumerate}	
如果想了解这类函数名的具体信息,可以查看我们提供的cocos.txt文件
(位于本调研文档所在目录中的result/MoonWarriors文件夹中)。
这类函数应该不止这些,还有一些未包含cocos关键字,这里没有列举出来。


\subsection{Lua VM提供的API}
\label{sec:so:lualib}

Lua VM提供的API的实现源码位于cocos2d-x-3.3/external/lua/lua目录下,
这些函数是Lua 提供的接口,帮助lua成为一门小巧方便的嵌入语言。
举例如下:
\begin{lstlisting}
	luaL_ref
	luaL_unref
	luaL_checkany
	luaopen_table
	luaopen_ffi
	luaopen_os
	luaopen_io
	...
\end{lstlisting}
该类函数名形式为:luaL\_*或luaopen\_*,
在libgame.so中出现的这类函数的函数名信息可参见luaopen\_prefix.txt
和lua\_prefix.txt文件
(位于本调研文档所在目录中的result/MoonWarriors文件夹中)。

\subsection{系统级JNI本地函数}
\label{sec:so:sysjnilib}
系统级JNI本地函数是cocos2d-x中针对android系统中的JNI函数给出的本地函数实现,
这类函数的实现源码位于cocos2d-x-3.3/cocos/platform/android目录下。
利用objdump查看libgame.so中可得到相关的这类函数的部分符号信息如下:
\begin{lstlisting}
001f12d0 g    DF .text  00000068 openKeyboardJNI
001f1338 g    DF .text  00000068 closeKeyboardJNI
0021e540 g    DF .text  00000054 endJNI
0021e0d4 g    DF .text  00000080 preloadBackgroundMusicJNI
0021e154 g    DF .text  00000088 playBackgroundMusicJNI


001efe48  w   DF .text  00000018 _ZN7_JNIEnv12NewStringUTFEPKc
001f007c g    DF .text  000000f0 _Z21showEditTextDialogJNIPKcS0_iiiiPFvS0_PvES1_
001f01cc g    DF .text  000000ac _Z17getPackageNameJNIv
001f0598 g    DF .text  0000009c _Z16getBoolForKeyJNIPKcb
001f0634 g    DF .text  00000094 _Z19getIntegerForKeyJNIPKci
001f06c8 g    DF .text  0000009c _Z17getFloatForKeyJNIPKcf
001f0764 g    DF .text  0000009c _Z18getDoubleForKeyJNIPKcd
001f0800 g    DF .text  00000130 _Z18getStringForKeyJNIPKcS0_
001f0930 g    DF .text  0000008c _Z16setBoolForKeyJNIPKcb
001f09bc g    DF .text  0000008c _Z19setIntegerForKeyJNIPKci
001f0a48 g    DF .text  00000094 _Z17setFloatForKeyJNIPKcf
001f0adc g    DF .text  00000090 _Z18setDoubleForKeyJNIPKcd
001f0b6c g    DF .text  000000a8 _Z18setStringForKeyJNIPKcS0_
001efe10  w   DF .text  00000038 _ZN7_JNIEnv22CallStaticDoubleMethodEP7_jclassP10\_jmethodIDz
0018cf44 g    DF .text  00000018 JNI_OnLoad
\end{lstlisting}

上面的符号可以进一步分成如下3类:
\begin{enumerate}
\item 以JNI结尾的函数,这些函数我们目前推测是由cocos2d-x引擎对
android平台提供的Java接口的C++/C实现,并且是基于JNI机制实现的。
\item 经过编译器为去除面向对象特征而转换得到的函数,
暂时还不清楚这些函数的具体作用。例如:
\begin{lstlisting}
_Z19terminateProcessJNIv
_Z22enableAccelerometerJNIv
...
\end{lstlisting}
\item JNI\_Onload:
当Android的VM(Virtual Machine)执行到应用软件里的 
System.loadLibrary(``libgame.so'')函数时,
首先会去执行应用软件的动态链接库libgame.so里的JNI\_OnLoad()函数。
它的用途如下:

1) 告诉VM此C组件使用哪一个JNI版本。如果你的*.so文件没有提供JNI\_OnLoad() 函数,
    VM会默认该*.so是使用最老的JNI 1.1版本。由于新版的JNI做了许多扩充,
    如果需要使用JNI的新版功能,例如JNI 1.4的 java.nio.ByteBuffer, 
    就必须借由JNI\_OnLoad()函数来告知VM。

2) 由于VM执行到System.loadLibrary()函数时,就会立即先呼叫JNI\_OnLoad(),
    所以C 组件的开发者可以借由JNI\_OnLoad()来进行C组件内的初期值之设定(Initialization)。\\
\end{enumerate}

下面我们仅以那些以JNI结尾的函数为例来进行进一步的分析。

\subsubsection{以JNI结尾的函数的具体分析}
首先,我们将这些函数全部罗列如下:
\begin{lstlisting}
getDPIJNI
openKeyboardJNI
closeKeyboardJNI
endJNI
preloadBackgroundMusicJNI
playBackgroundMusicJNI
stopBackgroundMusicJNI
pauseBackgroundMusicJNI
resumeBackgroundMusicJNI
rewindBackgroundMusicJNI
isBackgroundMusicPlayingJNI
getBackgroundMusicVolumeJNI
setBackgroundMusicVolumeJNI
getEffectsVolumeJNI
setEffectsVolumeJNI
playEffectJNI
stopEffectJNI
preloadEffectJNI
unloadEffectJNI
pauseEffectJNI
pauseAllEffectsJNI
resumeEffectJNI
resumeAllEffectsJNI
\end{lstlisting}

通过对.so利用objudmp -D反汇编,我们可以查看到libgame.so文件导出表的
所有汇编代码。
以playBackgroundMusicJNI为例子,
为了得到该函数所调用的函数,
从反汇编得到的汇编代码中,
我们只需查看其中的跳转指令。
最终,通过查看被调用函数的汇编代码及其跳转指令,
我们得到了函数playBackgroundMusicJNI调用关系如下:

\begin{itemize}
\item playBackgroundMusicJNI
  	\begin{itemize}
  	\item \verb|_ZN13CocosDenshion17SimpleAudioEngine14stopAllEffectsEv|
		\begin{itemize}
  		\item \verb|_ZN7cocos2d7CCPointC1Eff|
   		\item \verb|_ZN7cocos2d6CCSizeC1Eff|
   		\item \verb|_ZN7cocos2d6CCRect7setRectEffff|
  		\item \verb|_ZN13CocosDenshion17SimpleAudioEngine14stopAllEffectsEv|:
		即调用自身
		\item \verb|<android_setCpuArm+0x94c>|
		\item \verb|_ZN7cocos2d9JniHelper9getJavaVMEv|
  			\begin{itemize}
 			\item \verb|pthread_key_create@plt|
  			\item \verb|__android_log_print@plt|
  			\item \verb|pthread_getspecific@plt|
  			\item \verb|_ZN7cocos2d9JniHelper9getJavaVMEv| 自身递归
  			\end{itemize}
		\item \verb|__android_log_print@plt|
		\item 其他一些\verb|xxx@plt|函数
		\end{itemize}
   	\item \verb|_ZN7_JNIEnv14DeleteLocalRefEP8_jobject|
   	\item \verb|_ZN7_JNIEnv20Cal33lStaticVoidMethodEP7_jclassP10_jmethodIDz|
  		\begin{itemize}
  		\item ......
  		\end{itemize}
	\end{itemize}
\end{itemize}

总的来说,函数playBackgroundMusicJNI会调用如下三个函数和其自身。
\begin{lstlisting}
1._ZN13CocosDenshion17SimpleAudioEngine14stopAllEffectsEv
2._ZN7_JNIEnv20Cal33lStaticVoidMethodEP7_jclassP10_jmethodID
3._ZN7_JNIEnv14DeleteLocalRefEP8_jobject
\end{lstlisting}

其中1、2、3存在严格的顺序关系,即会先调用1,然后2,然后才是3;
而对自身的调用会被插入第1个函数与第2个函数之间,
或第2个函数与第3个函数之间或第3个函数以后。

深入函数1.\_ZN13CocosDenshion17SimpleAudioEngine14stopAllEffectsE,
我们发现该函数
调用的函数
包括如下5个函数以及其自身。
\begin{lstlisting}
_ZN7cocos2d7CCPointC1Eff
_ZN7cocos2d6CCSizeC1Eff
_ZN7cocos2d6CCRect7setRectEffff>
android_setCpuArm+0x94c
_ZN7cocos2d9JniHelper9getJavaVMEv
\end{lstlisting}

同样的,我们分析5.\_ZN7cocos2d9JniHelper9getJavaVMEv,可以发现 
该函数会对如下函数及其自身进行调用。

\begin{lstlisting}
pthread_key_create@plt
__android_log_print@plt
pthread_getspecific@plt
\end{lstlisting}

\subsection{C/C++层与Lua之间的绑定函数}
\label{sec:so:bindlib}
这些函数名的形式为tolua\_XXX.这些函数是用于将C/C++写的函数
自动注册到Lua VM中, 供Lua脚本语言调用。
这些函数的实现源码位于
cocos2d-x-3.3/external/lua/tolua目录下。

而针对这类函数,
我们也许可以在保持接口不变的情况下,
从逃逸分析的角度给出优化方法。
之所以认为这是一种可能的优化策略, 
主要是因为在那些用Lua编写的
代码(基于cocos2d-x引擎)中存在对C++层对象的调用, 
而这类函数会通过malloc将其放在堆上。

tolua\_XXX函数例如:\\
\begin{lstlisting}
tolua_isnoobj
tolua_typename
tolua_classevents
\end{lstlisting}

\subsection{应用级JNI本地函数}
\label{sec:so:appjnilib}
这类函数是游戏软件中本地函数的具体实现,
即是游戏软件的dex文件中具有native属性的自定义方法的具体实现。
而且我们暂时估计这些方法是一些由cocos2d-x提供的、
用于cocos2d-x与android平台建立联系的接口函数,
所有与引擎相关的native函数或者
第三方提供的sqlite,web,socket,ssl,openssl等native函数均通过这些
接口函数,最终作用于android平台。

这类函数位于cocos2d-x-3.3/cocos/platform/android目录下。

以我们分析的case为例,在dex文件中,有如下native function:
\begin{lstlisting}
onSensorChanged
nativeInitBitmapDC
nativeSetTextureInfo
nativeSetApkPath
callLuaFunctionWithString
callLuaGlobalFunctionWithString
releaseLuaFunction
retainLuaFunction
nativeDeleteBackward
nativeGetContentText
nativeInit
nativeInsertText
nativeKeyDown
nativeOnPause
nativeOnResume
nativeRender
nativeTouchesBegin
nativeTouchesCancel
nativeTouchesEnd
nativeTouchesMove
\end{lstlisting}

其中的nativeDeleteBackward函数对应的C++层函数为:

Java\_org\_cocos2dx\_lib\_Cocos2dxRenderer\_nativeDeleteBackward
该函数位于cocos2d-x/cocos/platform/android/jni目录下的\\
Java\_org\_cocos2dx\_lib\_Cocos2dxRenderer.cpp文件中

更直观的展示如下:

通过命令dexdump -d xxx.apk输出的文件中包含如下一个native method的信息
\begin{lstlisting}
    #2              : (in Lorg/cocos2dx/lib/Cocos2dxRenderer;)
      name          : 'nativeDeleteBackward'
      type          : '()V'
      access        : 0x010a (PRIVATE STATIC NATIVE)
      code          : (none)
\end{lstlisting}

通过如下命令:
\begin{lstlisting}
objdump -T libgame.so |grep "Java_org_cocos2dx_lib_Cocos2dxRenderer_nativeDeleteBackward"
\end{lstlisting}

可以输出
\begin{lstlisting}
001f0d48 <Java_org_cocos2dx_lib_Cocos2dxRenderer_nativeDeleteBackward>:
\end{lstlisting}

因此要使用的native function需要在.so文件和.dex文件中一一对应。

\subsection{总结}
\label{sec:so:conclusion}
通过以上对.so文件中的符号表进行分类,我们推测,
如果要对基于cocos2d-x的游戏进行优化,
可以只通过对Lua VM做优化,保持其对cocos2d-x的使用接口不变。
对Lua VM进行优化可能存在多种方案;
当然基于整体cocos2d-x游戏的优化
也未必一定是针对lua vm的优化,一切暂时还是未知。

另一方面,
为了将优化后的Lua VM能快速替换游戏软件中所使用的Lua VM版本,
我们可以尝试从一个或若干个通过NDK编译得到的.so单独剥离出liblua.so,
然后替换为我们修改Lua vm得到的那份新的liblua.so。
这里提到的liblua.so指的都是Lua vm编译得到的.so文件,只是可能需要进行修改。
而对其进行剥离也可能会在我们的考虑之内,甚至可能成为要开展的工作之一。


