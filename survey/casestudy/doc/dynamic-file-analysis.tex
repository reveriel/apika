\section{基于cocos2d-x 3.3的游戏软件中动态链接库分析}
\label{soanalysis}
我们以MoonWarriors软件包含的动态链接库libgame.so为例,
对该动态链接库包含的函数等进行深入分析。
我们用objdump命令,从libgame.so中导出符号表。
符号表中的信息可以参见\url{https://ring0.me/2014/12/linkers-loaders-library-notes/}文章最后的表格\\
我们将其中的函数分成如下五类,依次在下面各小节中分别说明:
\begin{enumerate}
\item \hei{游戏引擎库函数}:
	指由cocos2d-x引擎提供的游戏相关的库所包含的函数,
	参见第\ref{sec:so:cocolib}节;
\item \hei{Lua运行时函数}:
	Lua运行时环境(Lua虚拟机)提供的库函数,
	参见第\ref{sec:so:lualib}节;
\item \hei{系统级JNI本地函数}:
	cocos2d-x中针对android系统中的JNI函数给出的本地函数实现,
	参见第\ref{sec:so:sysjnilib}节;
\item \hei{C++与Lua之间的绑定函数}:
	cocos2d-x用于方便沟通C/C++层与Lua vm之间绑定(binding)所使用的、
	函数名以tolua开头的函数(一个自动bind工具),
	参见第\ref{sec:so:bindlib}节;
\item \hei{应用级JNI本地函数}:
	以Java\_org\_cocos2dx\_lib打头的函数,
	这些是针对cocos2d以及游戏应用的JNI函数给出的本地函数实现,
	参见第\ref{sec:so:appjnilib}节;
\end{enumerate}

\subsection{cocos2d-x游戏库函数}
\label{sec:so:cocolib}

\begin{table}[H]
\caption{\hei{Properties of Different Paging Modes}}
\begin{tabular}{|c|c|c|c|}
\hline 类别 & 该类符号的数目 & \% & .so中所有的符号数目\\
\hline cocos2d-x游戏库函数& 8431 & 0.52690 & 16431  \\
\hline  lua vm 提供的API & 74 & 0.00450 & 16431 \\
\hline C/C++与lua之间的绑定函数 & 79 &  0.004808 & 16431 \\
\hline 与JNI相关的函数(不含接口JNI) & 69 & 0.00420 & 16431 \\
\hline 接口JNI & 21 & 0.00128 & 16431 \\
\hline
\end{tabular}
\end{table}

首先根据objdump -T libgame.so|grep ".text" > \_text.txt 命令得到的文件\_text.txt,\\
此文件中的每一行的最后一项均表示一个游戏在运行过程中可能用到的函数名。

然后我们查看该文件,发现共有16431行,也就是说游戏运行过程中可能用到16431个函数。\\
而通过:\\
cat \_text.txt | grep  "\_Z" > \_z.txt\\
cat \_z.txt | grep "cocos2dx" > cocos2dx.txt\\
通过查看cocos2dx.txt,发现共有8431行,cocos2d-x游戏库提供了大概8431个函数,这些
函数都是经过编译器去除面向对象特性的函数。
 
e.g.:\\
      \_ZTSN7cocos2d9extension9CCBReaderE\\
      \_ZN7cocos2d10CCDirector13setProjectionENS\_20ccDirectorProjectionE\\
      \_ZTVN7cocos2d8CCActionE\\
      \_ZTIN7cocos2d8CCLiquidE\\
      \_ZNK7cocos2d13CCDictElement9getStrKeyEv\\
      \_ZN7\_JNIEnv16CallObjectMethodEP8\_jobjectP10\_jmethodIDz\\
这些函数都有一个共同特点:\\
以\_ZN7\_JNIEnv16CallObjectMethodEP8\_jobjectP10\_jmethodIDz 为例,它就是编译器将名字空间JNIEnv中
的CallObjectMethod(unsigned char,\_jobject* ,\_jmethod* )函数去除面向对象特征转换后得到的,
能唯一标识的符号名字.\\
\_Z 是所有改名后的符号的开始符号,\\
N表示namespace,在这里指的就是\_JNIEnv 名字空间,\\
在接下去CallObectMethod表示的就是原来的那个函数名,\\
再接下去就是函数参数列表中的参数类型,\\
E 表示unsigned char\\
P 表示指针,由于其后是\_jobject ,故此指针为\_jobject *\\
同理有指针\_jmethodIDz *.\\
	
如果想亲自查看第一类函数名的具体信息,请查看我们提供的cocos.txt文件.(位于android-work/survey/casestudy/result目录相应游戏目录下),第一类函数应该不止这些,还有一些未包含cocos关键字\\

\subsection{lua vm提供的API}
\label{sec:so:lualib}

该类符号表形式如下所示:\\
e.g.:\\
	luaL\_ref\\
	luaL\_unref\\
	luaL\_checkany\\
	luaopen\_table\\
	luaopen\_ffi\\
	luaopen\_os\\
	luaopen\_io\\
	...\\
一言以概之,其形式为:luaL\_(.)+ luaopen\_(.)+.
可以参见luaopen\_prefix.txt  lua\_prefix.txt文件.(位于android-work/survey/casestudy/result目录相应游戏目录下).

\subsection{系统级JNI本地函数}
\label{sec:so:sysjnilib}
\begin{lstlisting}
001f12d0 g    DF .text  00000068 openKeyboardJNI
001f1338 g    DF .text  00000068 closeKeyboardJNI
0021e540 g    DF .text  00000054 endJNI
0021e0d4 g    DF .text  00000080 preloadBackgroundMusicJNI
0021e154 g    DF .text  00000088 playBackgroundMusicJNI


001efe48  w   DF .text  00000018 _ZN7_JNIEnv12NewStringUTFEPKc
001f007c g    DF .text  000000f0 _Z21showEditTextDialogJNIPKcS0_iiiiPFvS0_PvES1_
001f01cc g    DF .text  000000ac _Z17getPackageNameJNIv
001f0598 g    DF .text  0000009c _Z16getBoolForKeyJNIPKcb
001f0634 g    DF .text  00000094 _Z19getIntegerForKeyJNIPKci
001f06c8 g    DF .text  0000009c _Z17getFloatForKeyJNIPKcf
001f0764 g    DF .text  0000009c _Z18getDoubleForKeyJNIPKcd
001f0800 g    DF .text  00000130 _Z18getStringForKeyJNIPKcS0_
001f0930 g    DF .text  0000008c _Z16setBoolForKeyJNIPKcb
001f09bc g    DF .text  0000008c _Z19setIntegerForKeyJNIPKci
001f0a48 g    DF .text  00000094 _Z17setFloatForKeyJNIPKcf
001f0adc g    DF .text  00000090 _Z18setDoubleForKeyJNIPKcd
001f0b6c g    DF .text  000000a8 _Z18setStringForKeyJNIPKcS0_
001efe10  w   DF .text  00000038 _ZN7_JNIEnv22CallStaticDoubleMethodEP7_jclassP10\_jmethodIDz
0018cf44 g    DF .text  00000018 JNI_OnLoad
\end{lstlisting}

上面的符号可以进一步分成如下3类:
\begin{itemize}
\item 以JNI结尾的函数,这些函数我们目前推测是由android平台提供的Java接口的C++/C实现,并且是基于JNI机制实现的。
\begin{lstlisting}
JNI_OnLoad
getDPIJNI
openKeyboardJNI
closeKeyboardJNI
endJNI
preloadBackgroundMusicJNI
playBackgroundMusicJNI
stopBackgroundMusicJNI
pauseBackgroundMusicJNI
resumeBackgroundMusicJNI
rewindBackgroundMusicJNI
isBackgroundMusicPlayingJNI
getBackgroundMusicVolumeJNI
setBackgroundMusicVolumeJNI
getEffectsVolumeJNI
setEffectsVolumeJNI
playEffectJNI
stopEffectJNI
preloadEffectJNI
unloadEffectJNI
pauseEffectJNI
pauseAllEffectsJNI
resumeEffectJNI
resumeAllEffectsJNI
\end{lstlisting}
\item 经过编译器为去除面向对象特征而转换得到的函数,暂时还不清楚这些函数的具体作用
\begin{lstlisting}
\_Z19terminateProcessJNIv
\_Z22enableAccelerometerJNIv
\_Z23disableAccelerometerJNIv
\_Z27setAccelerometerIntervalJNIf
\_Z21getCurrentLanguageJNIv
\_ZN7\_JNIEnv20CallStaticVoidMethodEP7\_jclassP10\_jmethodIDz
\_Z13showDialogJNIPKcS0\_
\_Z19getFileDirectoryJNIv
\_Z20inDirectoryExistsJNIPKc
\_ZN7\_JNIEnv22CallStaticObjectMethodEP7\_jclassP10\_jmethodIDz
\_ZN7\_JNIEnv19CallStaticIntMethodEP7\_jclassP10\_jmethodIDz
\_ZN7\_JNIEnv14DeleteLocalRefEP8\_jobject
\_ZN7\_JNIEnv23CallStaticBooleanMethodEP7\_jclassP10\_jmethodIDz
\_ZN7\_JNIEnv21CallStaticFloatMethodEP7\_jclassP10\_jmethodIDz
\_ZN7\_JNIEnv12NewStringUTFEPKc
\_Z21showEditTextDialogJNIPKcS0\_iiiiPFvS0\_PvES1\_
\_Z17getPackageNameJNIv
\_Z16getBoolForKeyJNIPKcb
\_Z19getIntegerForKeyJNIPKci
\_Z17getFloatForKeyJNIPKcf
\_Z18getDoubleForKeyJNIPKcd
\_Z18getStringForKeyJNIPKcS0\_
\_Z16setBoolForKeyJNIPKcb
\_Z19setIntegerForKeyJNIPKci
\_Z17setFloatForKeyJNIPKcf
\_Z18setDoubleForKeyJNIPKcd
\_Z18setStringForKeyJNIPKcS0\_
\_ZN7cocos2d9JniHelper10getClassIDEPKcP7\_JNIEnv
\end{lstlisting}
\item JNI\_Onload
\end{itemize}

\subsubsection{以JNI结尾的函数}
通过对.so利用objudmp -D反汇编,我们可以察看到由libgame.so文件导出表的所有汇编代码。

以playBackgroundMusicJNI为例子,从反汇编得到的汇编代码可分析得到
如下的函数调用关系:
\begin{itemize}
\item playBackgroundMusicJNI
  	\begin{itemize}
  	\item \verb|_ZN13CocosDenshion17SimpleAudioEngine14stopAllEffectsEv|
		\begin{itemize}
  		\item \verb|_ZN7cocos2d7CCPointC1Eff|
   		\item \verb|_ZN7cocos2d6CCSizeC1Eff|
   		\item \verb|_ZN7cocos2d6CCRect7setRectEffff|
  		\item \verb|_ZN13CocosDenshion17SimpleAudioEngine14stopAllEffectsEv| 
		自身递归
%	\end{itemize}
%%  \end{itemize}
		\item \verb|<android_setCpuArm+0x94c>|
		\item \verb|_ZN7cocos2d9JniHelper9getJavaVMEv|
  			\begin{itemize}
 			\item \verb|pthread_key_create@plt|
  			\item \verb|__android_log_print@plt|
  			\item \verb|pthread\_getspecific@plt|
  			\item \verb|_ZN7cocos2d9JniHelper9getJavaVMEv| 自身递归
  			\end{itemize}
		\item \verb|__android_log_print@plt|
		\item 其他一些\verb|xxx@plt|函数
		\end{itemize}
   	\item \verb|_ZN7_JNIEnv14DeleteLocalRefEP8_jobject|
   	\item \verb|_ZN7_JNIEnv20Cal33lStaticVoidMethodEP7_jclassP10_jmethodIDz|
  		\begin{itemize}
  		\item ......
  		\end{itemize}
	\end{itemize}
\end{itemize}
详细说明如下:\\
该函数会调用如下三个函数\\
\begin{lstlisting}
 555088   21e17c:       ebffff53        bl      21ded0 <_ZN13CocosDenshion17SimpleAudioEngine14stopAllEffectsEv+0x98>
 555101   21e1b0:       ebff3da5        bl      1ed84c <_ZN7_JNIEnv20Cal33lStaticVoidMethodEP7_jclassP10_jmethodIDz>
 555104   21e1bc:       ebff46f1        bl      1efd88 <_ZN7_JNIEnv14DeleteLocalRefEP8_jobject>
\end{lstlisting}
 4.  对自身的调用\\
其中1 2 3存在严格的序关系,即会先调用1,然后2,然后才是3,而对自身的调用会被插入1,2或2,3或3,以后。\\
其实如果继续观察如何调用该形式的其他函数,也是如上所示流程,它们的基本特点是一样的\\

1.<\_ZN13CocosDenshion17SimpleAudioEngine14stopAllEffectsEv>的形式是:不断的递归,同时还有对如下函数的调用:
\begin{lstlisting}
 430181 001aa5e4 <_ZN7cocos2d7CCPointC1Eff>:
 430182   1aa5e4:       e92d4800        push    {fp, lr}
 430183   1aa5e8:       e5801000        str     r1, [r0]
 430184   1aa5ec:       e28db004        add     fp, sp, #4
 430185   1aa5f0:       e5802004        str     r2, [r0, #4]
 430186   1aa5f4:       e8bd8800        pop     {fp, pc}

 430598 001aabe0 <_ZN7cocos2d6CCSizeC1Eff>:
 430599   1aabe0:       e92d4800        push    {fp, lr}
 430600   1aabe4:       e5801000        str     r1, [r0]
 430601   1aabe8:       e28db004        add     fp, sp, #4
 430602   1aabec:       e5802004        str     r2, [r0, #4]
 430603   1aabf0:       e8bd8800        pop     {fp, pc}


 430799 001aaeac <_ZN7cocos2d6CCRect7setRectEffff>:
 430800   1aaeac:       e92d4800        push    {fp, lr}
 430801   1aaeb0:       e28db004        add     fp, sp, #4
 430802   1aaeb4:       e5803008        str     r3, [r0, #8]
 430803   1aaeb8:       e59b3004        ldr     r3, [fp, #4]
 430804   1aaebc:       e5801000        str     r1, [r0]
 430805   1aaec0:       e5802004        str     r2, [r0, #4]
 430806   1aaec4:       e580300c        str     r3, [r0, #12]
 430807   1aaec8:       e8bd8800        pop     {fp, pc}
\end{lstlisting}

\verb|<android_setCpuArm+0x94c>|  

\verb|<_ZN7cocos2d9JniHelper9getJavaVMEv>| 
该函数会对\verb|<pthread_key_create@plt>|、
\verb|<__android_log_print@plt>|、
\verb|<pthread_getspecific@plt>|
以及其自身\verb|_ZN7cocos2d9JniHelper9getJavaVMEv|调用.

\verb|0018bd10 <__android_log_print@plt>|

其他一些\verb|xxx@plt|的函数

\verb|555104   21e1bc:       ebff46f1        bl      1efd88|
 
\verb|<_ZN7_JNIEnv14DeleteLocalRefEP8_jobject>|\\

该函数只有一个跳转:\\
\begin{lstlisting}
 001efd88 <_ZN7_JNIEnv14DeleteLocalRefEP8_jobject>:
         1efd88:       e92d4800        push    {fp, lr}
         1efd8c:       e5903000        ldr     r3, [r0]
         1efd90:       e28db004        add     fp, sp, #4
         1efd94:       e593305c        ldr     r3, [r3, #92]   ; 0x5c
         1efd98:       e12fff33        blx     r3
         1efd9c:       e8bd8800        pop     {fp, pc}
\end{lstlisting}

\subsection{C/C++层与Lua之间的绑定函数}
\label{sec:so:bindlib}
此类函数的形式为tolua\_XXX .

这些函数是用于将C/C++写的函数,自动注册到lua vm中,供lua脚本语言调用。

而针对这里,我们也许可以在保持接口不变的情况下,从逃逸分析的角度给出优化方法。

这些函数例如:\\
\begin{lstlisting}
tolua\_isnoobj
tolua\_typename
tolua\_classevents
\end{lstlisting}

\subsection{应用级JNI本地函数}
\label{sec:so:appjnilib}
这类函数是cocos2d-x直接面向android平台的native function,是dex文件中的native属性的方法的具体实现。\\
而且我们暂时估计这些方法是一些cocos2d-x提供的用于cocos2d-x与android平台建立联系的接口函数,所有的引擎相关的native 函数或者第三方提供的sqlite,web,socket,ssl,openssl等native function均通过这些接口函数,最终作用于android平台\\

以我们分析的case为例,在dex文件中,有如下native function:\\
\begin{lstlisting}
onSensorChanged:(FFFJ)V
nativeInitBitmapDC:(II[B)V
nativeSetTextureInfo:(II[BI)V
nativeSetApkPath:(Ljava/lang/String;)V
callLuaFunctionWithString:(ILjava/lang/String;)I
callLuaGlobalFunctionWithString:
releaseLuaFunction:(I)I
retainLuaFunction:(I)I
nativeDeleteBackward:()V
nativeGetContentText
nativeInit
nativeInsertText
nativeKeyDown
nativeOnPause
nativeOnResume
nativeRender
nativeTouchesBegin
nativeTouchesCancel
nativeTouchesEnd
nativeTouchesMove
\end{lstlisting}

其中的nativeDeleteBackward函数对应的C++层函数为:\\
\verb|Java_org_cocos2dx_lib_Cocos2dxRenderer_nativeDeleteBackward|
该函数位于\verb|/cocos2d-x/cocos/platform/android/jni|

目录下的Java\_org\_cocos2dx\_lib\_Cocos2dxRenderer.cpp|.

为了直观,再作如下说明:\\
通过命令dexdump -d xxx.apk输出的文件中包含如下一个native method的信息\\
\begin{lstlisting}
    #2              : (in Lorg/cocos2dx/lib/Cocos2dxRenderer;)
      name          : 'nativeDeleteBackward'
      type          : '()V'
      access        : 0x010a (PRIVATE STATIC NATIVE)
      code          : (none)
\end{lstlisting}

而\verb|Java_org_cocos2dx_lib_Cocos2dxRenderer_nativeDeleteBackward|
通过\\
\begin{lstlisting}
objdump -T libgame.so |grep "Java_org_cocos2dx_lib_Cocos2dxRenderer_nativeDeleteBackward"
\end{lstlisting}

可以输出
\begin{lstlisting}
001f0d48 <Java_org_cocos2dx_lib_Cocos2dxRenderer_nativeDeleteBackward>:
\end{lstlisting}
所以可知要使用的native function在.so .dex文件中将一一对应才行.\\
\subsection{总结}
\label{sec:so:conclusion}
通过以上对.so文件中的符号表进行分类,我们推测,应该说基本笃定如果要对基于cocos2d-x的游戏进行优化,可以只对lua vm做优化,保持其对cocos2d-x
的使用接口不变.本身对lua vm的优化就可能存在多种方案,当然基于整体cocos2d-x游戏的优化,也未必一定是针对lua vm的优化,一切暂时还处于未知。\\
此外,另一方面,也许我们也可能从一个或若干个通过NDK编译得到的.so单独剥离出liblua.so,然后替换为我们修改lua vm得到的那份新的liblua.so。\\
这里提到的liblua.so指的都是lua vm编译得到的.so文件,只是存在修改与未修改之分。而此剥离也许会在我们的思考之内,甚至可能成为考虑工作之一.\\
