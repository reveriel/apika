\subsection{dex文件反编译工具}
\label{dextools}
表~\ref{tab:dextools}列出了几种关于dex文件反编译的工具。
\begin{table}[htbp]
\caption{\label{tab:dextools}\hei{dex反编译工具}}
\begin{tabular}{|p{2cm}|p{9.5cm}|}
\hline
工具 & 描述\\
\hline
dex2jar & 将classes.dex反编译出jar文件,即apk的源程序文件的java字节码,然后可以使用jdgui查看dex2jar反编译出的jar文件\\
\hline
Dedexer & 可以读取dex格式的文件,生成一种类似于汇编语言的输出。这种输出与jasmin的输出相似,但包含的是Dalvik的字节码\\
\hline
androguard & androguard是基于python的,将apk文件中的dex文件,类,方法等都映射为python的对象\\\hline
dexdump & Android自身提供的一个dex文件的反编译工具\\
\hline
DexAnalyzer & 基于dexdump修改之后的dex文件解析工具,能够提取dex文件中的类和方法,并作相应的分类\\
\hline
\end{tabular}
\end{table}

\begin{itemize}
\item dex2jar\\
将classes.dex拷贝到dex2jar目录下,使用命令:

./dex2jar.sh classes.dex

得到classes\_dex2jar.jar文件

打开jdgui,导入文件classes\_dex2jar.jar,就可以查看反编译的jar文件。

\item Dedexer\\
Dedexer的网站:\url{http://dedexer.sourceforge.net}。

\item androguard\\
网站:\url{http://code.google.com/p/androguard/wiki/Installation}

安装完成后androguard目录下的所有py文件都是一个工具,然后可以利用这些工具分析apk文件获取相应的信息,比如androdd.py用来生成apk文件中每个类的方法的调用流程图;androlyze.py是一个强大的静态分析工具,它提供的一个独立的Shell环境来辅助分析人员执行分析工作。在终端提示符下执行“./androlyze.py -s”会进入androlyze 的Shell交互环境,分析人员可以在其中执行不同的命令,来满足不同情况下的分析需求。androlyze.py通过访问对象的字段与方法的方式来提供反馈结果,分析过程中可能会用到3个对象:apk文件对象、dex文件对象、分析结果对象。这3个对象是通过androlyze.py的Shell环境(以下简称Shell环境)来获取的。

\item dexdump\\
在android SDK中提供了一个名为dexdump的工具,可以打印DEX文件的信息,它的源代码在dalvik目录中。dexdump工具提供了一些命令行参数来帮助用户输出相应的dex文件信息,相应的参数如下:

-c: 验证DEX文件的校验和

-d :  反汇编代码段

-f : 显示文件头摘要

-h : 显示文件头详细信息

-i : 忽略文件校验

-l : 输出格式,可以是'plain'或者'xml'格式

-m :  打印出寄存器图

-t : 临时文件名称

\item DexAnalyzer\\
DexAnalyzer是我们在dexdump基础上做修改之后的dex文件解析工具,除了原有的dexdump所能输出的信息之外新增了以下信息的输出:

-a :表示要进行方法的输出

-s:输出程序自定义的方法以及类

-k:输出程序中的库方法及对应的类

-n:输出native方法及对应的类

\end{itemize}

