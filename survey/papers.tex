\documentclass{article}
\usepackage[colorlinks=true]{hyperref}

\begin{document}
About Sensor, GPS, WakeLock, or other energy related resources usage in Android Program.

\section{What is keeping my phone awake?: characterizing and detecting no-sleep
  energy bugs in smartphone apps}

\href{http://dl.acm.org/citation.cfm?id=2307661}{link}

Purdue University. MobiSys 2012, cited by 185


\textbf{Abstract:}

Despite their immense popularity in recent years, smartphones are and will
remain severely limited by their battery life. Preserving this critical resource
has driven smartphone OSes to undergo a paradigm shift in power management: by
default every component, including the CPU, stays off or in an idle state,
unless the app explicitly instructs the OS to keep it on! Such a policy
encumbers app developers to explicitly juggle power control APIs exported by the
OS to keep the components on, during their active use by the app and off
otherwise. The resulting power-encumbered programming unavoidably gives rise to
a new class of software energy bugs on smartphones called no-sleep bugs, which
arise from mis-handling power control APIs by apps or the framework and result
in significant and unexpected battery drainage.

This paper makes the first advances towards understanding and automatically
detecting software energy bugs on smartphones. It makes the following three
contributions: (1) we present the first comprehensive study of real world
no-sleep energy bug characteristics; (2) we propose the first automatic solution
to detect these bugs based on the classic reaching definitions dataflow analysis
algorithm; (3) we provide experimental data showing that our tool accurately
detected all 17 known instances of no-sleep bugs and found 34 new bugs in the 73
apps examined.

\textbf{Bug Patterns:}
\begin{itemize}
\item No-Sleep Code Path
  \begin{itemize}
  \item forget to release the wakelock throught the code
  \item some excution path doesn't have a release operation (\textbf{common})
  \item some higher level condition (like a deadlock) prevent the execution from
    reaching the release point
  \item do not undertand the life-cycle of Android processes (\textbf{most
      common})
  \end{itemize}
\item No-Sleep Race Condition, caused by race conditions in multi-threaded apps.
\item No-Sleep Dilation, hold wkaelock for longer than needed.
\end{itemize}

\textbf{Solution:
}dataflow analysis. % TODO



\section{Where has my battery gone? Finding sensor related energy black holes in
  smartphone applications}

\href{http://ieeexplore.ieee.org/abstract/document/6526708/}{link}

GreenDroid, HKUST, NanJing, PerCom 2013, Cited by 52

\href{http://www.cse.ust.hk/~andrewust/}{Yepang Liu}

based on Java PathFinder

assumption: sensory data should be efficiently used.

challenge: 1. Event driven programming paradigm; 2. how to analyze the
utilization of sensory data.

`` To address the first challenge, we derive an application execution model from
Android specifications. This model captures application-generic temporal rules
that specify calling relationships between event handlers. Enforcing these rules
would enable JPF to realistically execute an Android application. To address the
second challenge, we monitor an application’s execution, and perform dynamic
data flow tracking at a bytecode instruction level. ''

Contributions:
\begin{itemize}
\item We propose a runtime analysis technique to automatically analyze sensory
  data utilization at different states of an Android application.
\item We present an application execution model that captures
  application-generic temporal rules for event handler scheduling. This model is
  general enough to be used in other Android application analysis techniques.
\item We implement a prototype tool called GreenDroid. To the best of our
  knowledge, GreenDroid is the first JPF extension that is able to verify
  Android applications.
\item We evaluate GreenDroid using six popular Android applications. GreenDroid
  successfully located real energy inefficiency problems in four applications,
  and reported new problems for the remaining two.
\end{itemize}





\end{document}