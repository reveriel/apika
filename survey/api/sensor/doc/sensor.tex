\documentclass{article}


\title{Sensor API in Android}

\begin{document}

\section{Offical Documents for Sensors}

\subsection{SDK referecnce (for developers)}


\href{https://developer.android.com/guide/topics/sensors/sensors_overview.html#sensors-monitor} {Sensor~Overview}
\begin{itemize}
\item
  \href{https://developer.android.com/guide/topics/sensors/sensors_motion.html}
  {Motion Sensors}
  
  The accelerometer and gyroscope sensors are always hardware-based.
  
  The gravity, linear acceleration, rotation vector, significant motion, step
  counter, and step detector sensors are either hardware-based or
  software-based.
   
  
\item
  \href{https://developer.android.com/guide/topics/sensors/sensors_position.html}
  {Position Sensors}

  geomegnetic field sensor, orientation sensor and proximity sensor.

\item
  \href{https://developer.android.com/guide/topics/sensors/sensors_environment.html}
  {Environment Sensors}

  ambient temperature, light, pressure, relative humidity and temperature sensors.
\end{itemize}


Important Classes:

\begin{description}
\item [  \href{https://developer.android.com/reference/android/hardware/SensorManager.html}{SensorManager}]

  You can use this class to create an instance of the sensor service. This class
  provides various methods for accessing and listing sensors, registering and
  unregistering sensor event listeners, and acquiring orientation information.
  This class also provides several sensor constants that are used to report sensor
  accuracy, set data acquisition rates, and calibrate sensors.

\item [
  \href{https://developer.android.com/reference/android/hardware/Sensor.html}{Sensor}]

  You can use this class to create an instance of a specific sensor. This class
  provides various methods that let you determine a sensor's capabilities.

\item[\href{https://developer.android.com/reference/android/hardware/SensorEvent.html}{SensorEvent}]

  The system uses this class to create a sensor event object, which provides
  information about a sensor event. A sensor event object includes the following
  information: the raw sensor data, the type of sensor that generated the event,
  the accuracy of the data, and the timestamp for the event.

\item[\href{https://developer.android.com/reference/android/hardware/SensorEventListener.html}{SensorEventListener}]

  You can use this interface to create two callback methods that receive
  notifications (sensor events) when sensor values change or when sensor accuracy
  changes.
\end{description}


{
  \small
  \paragraph{note:}
  Always make sure to disable sensors you don't need especially when your
  activity is paused. Failling to do so can drain the battery in just a few hours.
  Note that the system will not desable sensors automatically when the screen
  turns off.

  It's also important to note that this example uses the onResume() and onPause()
  callback methods to register and unregister the sensor event listener. As a best
  practice you should always disable sensors you don't need, especially when your
  activity is paused. Failing to do so can drain the battery in just a few hours
  because some sensors have substantial power requirements and can use up battery
  power quickly. The system will not disable sensors automatically when the screen
  turns off.

  Unregister sensor listeners

  Be sure to unregister a sensor's listener
  when you are done using the sensor or when the sensor activity pauses. If a
  sensor listener is registered and its activity is paused, the sensor will
  continue to acquire data and use battery resources unless you unregister the
  sensor. The following code shows how to use the onPause() method to unregister a
  listener:

  Don't block the onSensorChanged() method

  Sensor data can change at a high rate, which means the system may call the
  onSensorChanged(SensorEvent) method quite often. As a best practice, you should
  do as little as possible within the onSensorChanged(SensorEvent) method so you
  don't block it. If your application requires you to do any data filtering or
  reduction of sensor data, you should perform that work outside of the
  onSensorChanged(SensorEvent) method.

  In the case of non-wake-up sensors, the events are only delivered while the
  Application Processor (AP) is not in suspend mode. See isWakeUpSensor() for more
  details. To ensure delivery of events from non-wake-up sensors even when the
  screen is OFF, the application registering to the sensor must hold a partial
  wake-lock to keep the AP awake, otherwise some events might be lost while the AP
  is asleep. Note that although events might be lost while the AP is asleep, the
  sensor will still consume power if it is not explicitly deactivated by the
  application.
}

\subsection{Documents targeted at manufacturers}

\href{https://source.android.com/devices/sensors/index.html}{Sensors}
\begin{itemize}
\item \href{https://source.android.com/devices/sensors/suspend-mode.html}
  {Suspend mode} (non-wake-up and wake-up sensors)
  
\item \href{https://source.android.com/devices/sensors/batching.html}
  {Batching} (FIFO)
\end{itemize}





\end{document}
