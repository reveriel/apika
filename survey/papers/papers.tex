\documentclass{article}
\usepackage[colorlinks=true]{hyperref}

\begin{document}
About Sensor, GPS, WakeLock, or other energy related resources usage in Android Program.

\section{What is keeping my phone awake?: characterizing and detecting no-sleep
  energy bugs in smartphone apps}

\href{http://dl.acm.org/citation.cfm?id=2307661}{link}

Purdue University, static, MobiSys 2012, cited by 185

Wake lock, static, ``the first to study energy bugs caused by wake lock
leakage''(ELITE),



\textbf{Abstract:}

Despite their immense popularity in recent years, smartphones are and will
remain severely limited by their battery life. Preserving this critical resource
has driven smartphone OSes to undergo a paradigm shift in power management: by
default every component, including the CPU, stays off or in an idle state,
unless the app explicitly instructs the OS to keep it on! Such a policy
encumbers app developers to explicitly juggle power control APIs exported by the
OS to keep the components on, during their active use by the app and off
otherwise. The resulting power-encumbered programming unavoidably gives rise to
a new class of software energy bugs on smartphones called no-sleep bugs, which
arise from mis-handling power control APIs by apps or the framework and result
in significant and unexpected battery drainage.

This paper makes the first advances towards understanding and automatically
detecting software energy bugs on smartphones. It makes the following three
contributions: (1) we present the first comprehensive study of real world
no-sleep energy bug characteristics; (2) we propose the first automatic solution
to detect these bugs based on the classic reaching definitions dataflow analysis
algorithm; (3) we provide experimental data showing that our tool accurately
detected all 17 known instances of no-sleep bugs and found 34 new bugs in the 73
apps examined.

\textbf{Bug Patterns:}
\begin{itemize}
\item No-Sleep Code Path
  \begin{itemize}
  \item forget to release the wakelock throught the code
  \item some excution path doesn't have a release operation (\textbf{common})
  \item some higher level condition (like a deadlock) prevent the execution from
    reaching the release point
  \item do not undertand the life-cycle of Android processes (\textbf{most
      common})
  \end{itemize}
\item No-Sleep Race Condition, caused by race conditions in multi-threaded apps.
\item No-Sleep Dilation, hold wkaelock for longer than needed.
\end{itemize}

\textbf{Solution:
}dataflow analysis. by reaching definition 

TO BE COMPLETED


\section{Where has my battery gone? Finding sensor related energy black holes in
  smartphone applications}

\href{http://ieeexplore.ieee.org/abstract/document/6526708/}{link}

GreenDroid, dynamic, HKUST, NJU, PerCom 2013, Cited by 52

\href{http://www.cse.ust.hk/~andrewust/}{Yepang Liu}

based on Java
PathFinder(\href{http://babelfish.arc.nasa.gov/trac/jpf/wiki}{link})
(a customizable virtual machine that enables the development of various
verification algorithms)

assumption: sensory data should be efficiently used.

challenge: 1. Event driven programming paradigm; 2. how to analyze the
utilization of sensory data.

`` To address the first challenge, we derive an application execution model from
Android specifications. This model captures application-generic temporal rules
that specify calling relationships between event handlers. Enforcing these rules
would enable JPF to realistically execute an Android application. To address the
second challenge, we monitor an application’s execution, and perform dynamic
data flow tracking at a bytecode instruction level. ''

Contributions:
\begin{itemize}
\item We propose a runtime analysis technique to automatically analyze sensory
  data utilization at different states of an Android application.
\item We present an application execution model that captures
  application-generic temporal rules for event handler scheduling. This model is
  general enough to be used in other Android application analysis techniques.
\item We implement a prototype tool called GreenDroid. To the best of our
  knowledge, GreenDroid is the first JPF extension that is able to verify
  Android applications.
\item We evaluate GreenDroid using six popular Android applications. GreenDroid
  successfully located real energy inefficiency problems in four applications,
  and reported new problems for the remaining two.
\end{itemize}


\section{Understanding and Detecting Wake Lock Misuses for Android Applications}

\href{http://dl.acm.org/citation.cfm?id=2950297}{link}

ELITE, static, HKUST, NJU, Yepang Liu, FSE 2016

assumption: wake lock should protect critical computational task.

solution: static method
\begin{enumerate}
\item summarize the effect about wakelock and task for each top level method
  (callback for event).
\item Generate all possible sequences for top level method.
\item sequences should be valid.(components' lifecycle; some callback must be
  registered first).
\item Check.
\end{enumerate}


\textbf{Related Work worth reading}


\textbf{Abstract}:

Wake locks are widely used in Android apps to protect critical computations from
being disrupted by device sleeping. Inappropriate use of wake locks often
seriously impacts user experience. However, little is known on how wake locks
are used in real-world Android apps and the impact of their misuses. To bridge
the gap, we conducted a large-scale empirical study on 44,736 commercial and 31
open-source Android apps. By automated program analysis and manual
investigation, we observed
\begin{itemize}
\item (1) common program points where wake locks are acquired and released,
\item (2) 13 types of critical computational tasks that are often protected by
  wake locks, and
\item (3) eight patterns of wake lock misuses that commonly cause functional and
  non-functional issues, only three of which had been studied by existing work.
\end{itemize}
Based on our findings, we designed a static
analysis technique, Elite, to detect two most common patterns of wake lock
misuses. Our experiments on real-world subjects showed that Elite is effective
and can outperform two state-of-the-art techniques.


\section{Diagnosing Energy Efficiency and Performance for Mobile Internetware
  Applications}

\href{http://sccpu2.cse.ust.hk/andrewust/files/ieeesoft15.pdf}{link}

HKUST, NJU, Yepang Liu, IEEE Software 2015


``Here, we discuss the challenges in diagnosing energy and performance bugs in
real-life Android applications. We hope to inspire further efforts to adequately
address these challenges. We also \textbf{review state-of-the-art diagnostic
  techniques and tools}. In particular, we offer results from our case study,
which applied a representative tool to popular commercial Android applications
and the Samsung Mobile Software Development Kit (SDK).''

\subsection{State-of-the-art Diagnosis and Tools}
Measurement and Estimation Techniques: measure or estimate app's energy
consumption and performance.
\begin{itemize}
\item vLens(fine-graned source code level), eProf, PowerTutor
\item Arvind Thiagarajan's framework for energy anlysis of rendering webpages
\item Mantis's, precise performance models for app, estimate execution time.
\end{itemize}

Event Profilers
\begin{itemize}
\item ARO (Application Resource Optimizer) monitors cross-layer
  interactions—such as user events at the application layer and network packets
  at the system layer—to disclose inefficient radio resource usage
\item AppInsight helps instrument smartphone applications’ binaries to identify
  long latency execution paths
\item Panappticon identies performance issues arising from inefficient platform
  code or problematic interactions among app.
\item SunCat logs the events in a test run and summarizes event repetition
  patterns to help developers understand and predict performance problems
\end{itemize}

Pattern-Based Analyzers
\begin{itemize}
\item dynamic analyzers. ADEL (Automated Detector of Energy Leaks) and
  GreenDroid Execute an application and track program-data transformation,
  propagation and consumption.
\item static.
  \begin{itemize}
  \item 
    Abhinav Pathak's work;
  \item Lint, a popular static analyzer in the Android Studio SDK, detects a
    range of energy and performance bugs.
  \item PerfChecker can detect eight patterns of energy and performance bugs and
    provide actionable diagnostic information
  \end{itemize}
\end{itemize}

End-User-Oriented Diagnosis(not for developers)
\begin{itemize}
\item eDoctor can correlate system and user events (such as conguration changes)
  to energy-heavy execution phases.
\item Carat shares the same goal as eDoctor but adopts a collaborative,
  big-data-driven approach.
\end{itemize}

\subsection{Disussion}
\begin{itemize}
\item we should check for neccessity, not only energy cost.
\item profilers can generate large profiles, visualization is desirable.(Maybe
  Machine Learning? gx)
\item pattern based methods should find root causes.
\end{itemize}


\section{Mining energy-greedy API usage patterns in Android apps: an empirical
  study}

\href{http://dl.acm.org/citation.cfm?id=2597085}{link}

Univ of Molise, Italy, College of William and Mary, USA, MSR 2014, cited by 67

hardware power monitor.

\textbf{Question}: is sensor and GPS really energy-greedy? TO BE ANSWERED

\textbf{Abstract}:

Energy consumption of mobile applications is nowadays a hot topic, given the
widespread use of mobile devices. The high demand for features and improved user
experience, given the available powerful hardware, tend to increase the apps’
energy consumption. However, excessive energy consumption in mobile apps could
also be a consequence of energy greedy hardware, bad programming practices, or
particular API usage patterns. We present the largest to date quantitative and
qualitative empirical investigation into the categories of API calls and usage
patterns that—in the context of the Android development framework—exhibit
particularly high energy consumption profiles. By using a hardware power
monitor, we measure energy consumption of method calls when executing typical
usage scenarios in 55 mobile apps from different domains. Based on the collected
data, we mine and analyze energy-greedy APIs and usage patterns. We zoom in and
discuss the cases where either the anomalous energy consumption is unavoidable
or where it is due to suboptimal usage or choice of APIs. Finally, we synthesize
our findings into actionable knowledge and recipes for developers on how to
reduce energy consumption while using certain categories of Android APIs and
patterns.

\section{Towards verifying android apps for the absence of no-sleep energy bugs}
\href{http://dl.acm.org/citation.cfm?id=2387872}{link} and
\href{https://www.usenix.org/conference/hotpower12/workshop-program/presentation/vekris}{presentation}
and
\href{https://www.usenix.org/system/files/conference/hotpower12/hotpower12-final27.pdf}{pdf}

Univ of California, San Diego, short paper, USENIX 2012, cited by 31

Wakelock, pricise inter-procedural data flow analysis framework. \emph{no-sleep}
bugs. use WALA. no code. 

Pollcy: (Component, Callback, Expected WakeLock state). 

Data Fact: the wakelock instances at this program point it may hold. the set
should be empty at certain program point(like \textbf{onPause} in Activity).

Asynchronous Calls: only explicit intents in which a single terget component is
specified.

Lifecycles: the exit ICFG node of a callback method is connected to every entry
CFG node of each of its successor method in the component’s lifecycle.


\textbf{Contributions}:
\begin{itemize}
\item First, we have studied the lifecycle of different kinds of components in
  Android applications, and \textbf{use this information to precisely define a
    set of resource management policies specifying the correct usage of wake
    locks.}
\item Second, compared to prior work we apply more precise analysis techniques
  for handling asynchronous calls, which are ubiquitous in Android.
\item Third, to evaluate our techniques we run our tool on 328 real world
  Android applications and verify that 145 of those are exempt from the specific
  kind of no-sleep bug (w.r.t. our policies).
\end{itemize}


\section{Static control-flow analysis of user-driven callbacks in Android
  applications}

\href{http://dl.acm.org/citation.cfm?id=2818768}{link}

Ohio State University, Atanas Rountev, ICSE 15, cited by 56


\section{WLCleaner: Reducing Energy Waste Caused by WakeLock Bugs at Runtime}

WLCleaner, runtime fix. DASC 2014,


\section{LEAKPOINT: Pinpointing the Causes of Memory Leaks}

LEAKPOINT, conventional software, ICSE 2010

\section{eDoctor Automatically Diagnosing Abnormal Battery Drain Issues on
  Smartphones.}

eDoctor, NSDI 2013

\section{Effective Interprocedural Resource Leak Detection}

Java, IBM, ICSE 2010

\section{Learning Resource Management Specifications in Smartphones}

ICPADS 2015

IMPORTANT, TO BE COMPLETED.

statically analyzes Android apps to mine resource management specifications
(i.e., the correct order of resource operations)




\end{document}
