\section{apk结构}
解压apk后,一般的可看到的目录结构如下:
\begin{table}[htbp]
\centering
\caption{\label{apk}apk组成}
\begin{tabular}{|c|c|}
\hline
文件或目录 & 作用\\
\hline
assets/ & 存放资源文件的目录\\
\hline
META-INF/ & 从java jar文件引入的描述包信息的目录\\
\hline
res/ & 存放资源文件的目录\\
\hline
libs/ & 如果存在的话,存放的是ndk编出来的so库\\
\hline
AndroidManifest.xml & 程序全局配置文件\\
\hline
classes.dex & 最终生成的dalvik字节码\\
\hline
resources.ars & 编译后的二进制资源文件\\
\hline
\end{tabular}
\end{table}

补充说明:\\
res和assets文件夹来存放不需要系统编译成二进制的文件,例如字体文件等,两者的区别:

1.assets:不会在R.java文件下生成相应的标记,assets文件夹可以自己创建文件夹,必须使用AssetsManager类进行访问,存放到这里的资源在运行打包的时候都会打入程序安装包中

2.res:会在R.java文件下生成标记,这里的资源会在运行打包操作的时候判断哪些被使用到了,没有被使用到的文件资源是不会打包到安装包中的
\section{dex文件解析}
在android SDK中提供了一个名为dexdump的工具,可以打印DEX文件的信息,它的源代码在dalvik目录中。dexdump工具提供了一些命令行参数来帮助用户输出相应的dex文件信息,相应的参数如下:

-c: 验证DEX文件的校验和

-d :  反汇编代码段

-f : 显示文件头摘要

-h : 显示文件头详细信息

-i : 忽略文件校验

-l : 输出格式,可以是'plain'或者'xml'格式

-m :  打印出寄存器图

-t : 临时文件名称

我们使用dexdump的-d选项对从网上下载过来Android游戏程序进行解析,发现得到的dalvik字节码文件过大,基本上都有几万行,而依据课题组的需求,对Android游戏程序关注的焦点应该在其调用的库方法上面,因此希望能够从dex文件中提取Android游戏程序调用的库方法有哪些,因此选择从dexdump入手,修改dexdump文件获取我们所需的信息。这里以两个游戏例子为例说明目前dex文件解析的结果。具体解析结果位于本目录下相应游戏文件夹中的文本文件,libclass.txt对应核心类,libmethod.txt对应库方法
\begin{itemize}
\item MoonWarriors\\
MoonWarriors是一个使用Cocos2d-x LUA开发的类似雷电战机的游戏Demo,源代码发布在Cocos2d-x官网的引擎示例当中:\url{http://cn.cocos2d-x.org/tutorial/show?id=2254}

分析MoonWarriors游戏程序调用的库方法得到有316个,分析其对应的核心类有87个 
 
\item Temple Run\\
神庙逃亡(Temple Run)是由 Imangi Studios开发的没有终点的动作类视频游戏,在Android平台采用统一的Unity游戏引擎。下载来源:\url{http://temple-run.cn.uptodown.com/android}

分析Temple Run游戏程序调用的库方法得到有2389个,分析其对应的核心类有412个
\end{itemize}
\redt{问题:}\\
1.Android核心库中主要有4类API:Java标准API(java包)、Java扩展API(javax包)、Android包以及企业和组织提供的Java类库(org包等)。从2个例子解析结果来看游戏软件中调用的库方法在这4类API都有覆盖,不可能对所有的核心库都做优化

2.对于Temp Run来说有2000多个库方法的调用,实际中可能有些游戏程序会比Temp Run更大,人工从几千个方法中做分析,总结哪些方法可以做优化是很困难的

3.因为我们有一个想法是探寻Android游戏软件中I/O角度的并行性,所以打算对dex文件解析的结果做进一步筛选,只提取输入输出相关的类方法,这个暂时还没有进行,目前还不能给出结论。希望能够了解商业应用中是否针对游戏软件中调用核心库方法有一些关注的焦点或者相关的内容

\section{cocos2d-x中lua相关的问题}
1.就目前基于cocos2d-x开发的游戏而言,是否有在linux+android studio上开发的呢?而新版的android studio在linux下存在各种问题,在导入工程后,居然没法将cocos2d-x游戏中的需要的一些文件包含进工程中。不知道是不是和gradle build system 有关。环境很重要,不然以后即使修改了代码也不能测试到底修改成功与否。我知道在android studio推出之前,是有linux+eclipse的,此eclipse当时也可以在\url{http://developer.android.com/sdk/index.html#Other}找到,但是自从android studio 推出后,貌似之前的google版本的eclipse就下架了。

作为例证:\url{http://www.cnblogs.com/lonkiss/archive/2012/11/17/2775440.html}该网页可以说明此事。

2.\redt{这个只是问老师的}是不是只有那写不是声明为static的函数才可能出现在.so文件的导出表中呢?

3.cocos2d-x中的lua解释器代码是否是从\url{http://www.lua.org/download.html}直接获取使用?或者是经修改之后放到cocos2d-x中使用?如果经过修改,不知道修改过哪些东西?

4.cocos2d-x中lua-bindings和lua解释器之间的接口是什么?对lua解释器的修改是否会牵涉到lua-bindings的修改呢?

5.quick-cocos2d-x 是基于 cocos2d-x + Lua,不知道现在quick-cocos2d-x和cocos2d-x+lua哪个受众更多?cocos2d-x中的bind方法似乎经常被第三方自己修改,不知道是不是这么一回事?
