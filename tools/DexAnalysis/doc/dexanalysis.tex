\section{DexAnalysis}
\subsection{设计需求}
Android应用程序包APK解压之后的目录结构如下:
\begin{table}[htbp]
\centering
\caption{\label{apk}apk组成}
\begin{tabular}{|c|c|}
\hline
文件或目录 & 作用\\
\hline
assets/ & 存放资源文件的目录\\
\hline
META-INF/ & 从java jar文件引入的描述包信息的目录\\
\hline
res/ & 存放资源文件的目录\\
\hline
libs/ & 如果存在的话,存放的是ndk编出来的so库\\
\hline
AndroidManifest.xml & 程序全局配置文件\\
\hline
classes.dex & 最终生成的dalvik字节码\\
\hline
resources.ars & 编译后的二进制资源文件\\
\hline
\end{tabular}
\end{table}

补充说明:\\
res和assets文件夹来存放不需要系统编译成二进制的文件,例如字体文件等,两者的区别:

1.assets:不会在R.java文件下生成相应的标记,assets文件夹可以自己创建文件夹,必须使用AssetsManager类进行访问,存放到这里的资源在运行打包的时候都会打入程序安装包中

2.res:会在R.java文件下生成标记,这里的资源会在运行打包操作的时候判断哪些被使用到了,没有被使用到的文件资源是不会打包到安装包中的

应用程序到Android上运行时是由dex文件开始的,所以我们需要了解classes.dex文件的结构。在android SDK中提供了一个名为dexdump的工具,可以打印DEX文件的信息,它的源代码在dalvik目录中。dexdump工具提供了一些命令行参数来帮助用户输出相应的dex文件信息,相应的参数如下:

-c: 验证DEX文件的校验和

-d :  反汇编代码段

-f : 显示文件头摘要

-h : 显示文件头详细信息

-i : 忽略文件校验

-l : 输出格式,可以是'plain'或者'xml'格式

-m :  打印出寄存器图

-t : 临时文件名称

但是分析Android应用程序的特征,依靠android自身提供dexdump工具解析dex文件,输出的dex字节码往往有几万行,很难从中提取重要的信息,因此我们选择对dexdump工具进行修改使其输出我们所需的信息。

\subsection{代码修改}
我们希望能够对Android游戏程序中的部分代码进行优化,但首先需要确定程序的哪部分能够进行优化,需要对这些应用程序进行案例分析。我们目前关注于游戏程序中的方法调用。因此对dexdump代码做出的修改,需要输出以下几类信息:

1.程序中调用了哪些库方法以及对应的类

2.程序中哪些方法属于自定义方法,以及自定义的类

3.在这些自定义的方法中哪些是native method以及对应的类

4.源程序中加载了哪些库,对应的是system.loadLibrary函数的参数信息

依据上面的需求,修改后的dexdump中我们添加了几个命令行参数,用于输出我们所需的信息

-a :表示要进行方法的输出

-s:输出程序自定义的方法以及类

-k:输出程序中的库方法及对应的类

-n:输出native方法及对应的类

\redt{补充:\\}
关于loadLibrary函数的参数信息的输出,因为在现有的dexdump相关的文件中对参数信息只能获得函数参数个数、大小以及类型的信息,还没有找到如何输出函数参数名的方法,以后若有发现会对本文档以及相关代码做修改。






