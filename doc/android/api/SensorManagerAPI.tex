
\subsection{How to get a wakeup or non-wakeup sensor}

\texttt{SensorManager} provides three ways to get a \texttt{Sensor} (not
dynamic)

\begin{lstlisting}
Sensor getDefaultSensor(int type)
Sensor getDefaultSensor(int type, boolean wakeup)
List<Sensor> getSensorList(int type)
\end{lstlisting}

此外class Sensor没有构造函数,其public method方法名
都是get开头的,也就是说获取sensor的相关信息,所以也不太可能
通过class Sensor的接口对sensor进行配置。下面详细介绍
三个函数的语义。
\par{
如果指定功能类型的wakeup/non-wakeup sensor
确实存在,那么
\begin{lstlisting}
Sensor getDefaultSensor(type,true/false)
\end{lstlisting}
该功能类型的wakeup/non-wakeup sensor引用,
否则返回null。
}
\par{
\begin{lstlisting}
List<Sensor> getSensorList(int type)
\end{lstlisting}
返回所有的指定功能类型的sensor,同时包括了
wakeup和non-wakeup sensor,程序员可能需要用
\begin{lstlisting}
Sensor.isWakeUpSensor(void)
\end{lstlisting}
方法确定返回的列表中的sensor是否是wakeup的。
}

\par{
根据
\href{https://source.android.com/devices/sensors/sensor-types.html}
     {Sensor-Types文档}
和
\href{https://developer.android.com/reference/android/hardware/SensorManager.html}{SensorManager文档}
calss SensorManager.getDefaultSensor(int type)
\begin{lstlisting}
Sensor getDefaultSensor(int type)
\end{lstlisting}
的行为由下面的内容描述,如果落在wakeup/non-wakeup类别中的
某一个功能类型的sensor不存在,那么函数返回null
\begin{itemize}
  \scriptsize
    \item non-wakeup
      \begin{itemize}
      \item SENSOR\_TYPE\_ACCELEROMETER
      \item SENSOR\_TYPE\_AMBIENT\_TEMPERATURE
      \item SENSOR\_TYPE\_MAGNETIC\_FIELD
      \item SENSOR\_TYPE\_GYROSCOPE
      \item SENSOR\_TYPE\_HEART\_RATE
      \item SENSOR\_TYPE\_LIGHT
      \item SENSOR\_TYPE\_PRESSURE
      \item SENSOR\_TYPE\_RELATIVE\_HUMIDITY
      \item SENSOR\_TYPE\_LINEAR\_ACCELERATION
      \item SENSOR\_TYPE\_STEP\_DETECTOR
      \item SENSOR\_TYPE\_STEP\_COUNTER
      \item SENSOR\_TYPE\_ROTATION\_VECTOR
      \item SENSOR\_TYPE\_GAME\_ROTATION\_VECTOR
      \item SENSOR\_TYPE\_GRAVITY
      \item SENSOR\_TYPE\_GEOMAGNETIC\_ROTATION\_VECTOR
      \item SENSOR\_TYPE\_ORIENTATION (deprecated)
      \item SENSOR\_TYPE\_GYROSCOPE\_UNCALIBRATED
      \item SENSOR\_TYPE\_MAGNETIC\_FIELD\_UNCALIBRATED
      \end{itemize}
    \item wakeup
      \begin{itemize}
      \item SENSOR\_TYPE\_PROXIMITY
      \item SENSOR\_TYPE\_SIGNIFICANT\_MOTION
      \item SENSOR\_TYPE\_TILT\_DETECTOR
      \item SENSOR\_TYPE\_WAKE\_GESTURE
      \item SENSOR\_TYPE\_PICK\_UP\_GESTURE
      \item SENSOR\_TYPE\_GLANCE\_GESTURE
      \end{itemize}
\end{itemize}

\subsection{Sensor Registration and Unregistration}
\subsubsection{Registration}
Class SensorManager中用于sensor registration的API有
\begin{lstlisting}
boolean registerListener(SensorEventListener listener, Sensor sensor,
                        int samplingPeriodUs)
boolean registerListener(SensorEventListener listener, Sensor sensor,
                        int samplingPeriodUs, int maxReportLatencyUs)
boolean registerListener(SensorEventListener listener, Sensor sensor,
                        int samplingPeriodUs, Handler handler)
boolean registerListener(SensorEventListener listener, Sensor sensor,
                        int samplingPeriodUs, int maxReportLatencyUs, Handler handler)
\end{lstlisting}
其中参数、返回值的含义是
\begin{itemize}
  \item listener: omitted
  \item sensor: omitted
  \item samplingPeriodUs:The rate sensor events are delivered at. 
    应该注意这个参数并不是指硬件的采样频率。此外这个参数对于Android System
    而言只是一个hint,实际events被deliver的频率可以快于或者慢于这个参数
    指定的频率(通常是快于)。实参的值可以是
      \begin{itemize}
        \item SENSOR\_DELAY\_NORMAL
        \item SENSOR\_DELAY\_UI
        \item SENSOR\_DELAY\_GAME
        \item SENSOR\_DELAY\_FASTEST
      \end{itemize}
      或者以microseconds为单位的期望的值。
  \item maxReportLatencyUs : 当 sensor event 到达后,最大的可以延迟该 event delivery
    的时间,单位是 microseconds,如果这个参数是0那么等价于调用
    registerListener ( listener, sensor, samplingPeriodUs ) 。
  \item handler: the handler the sensor event will be delivered to
  \item 返回值: true if the sensor is supported and successfully enabled
\end{itemize}

\subsubsection{Unregistration}
Class SensorManager 中用于 sensor registration 的 API 有
\begin{lstlisting}
void unregisterListener(SensorEventListener listener)
//unregisters a listener for all sensors
void unregisterListener(SensorEventListener listener, Sensor sensor)
//unregister a listener for the sensors with which it is registered
\end{lstlisting}

\subsubsection{TODO}
这里是一些文档中提到的good practice,或许能作为TODO
\begin{itemize}
  \item 文档提到不要将 registerListener 函数用于 one shot trigger sensor such as
SENSOR\_TYPE\_SIGNIFICANT\_MOTION。(sensor event 提交的方式被分为 continuous 和
one shot 两种 ( 不同功能类型不同 ),TODO)
  \item samplingPeriodUs 和 maxReportLatencyUs 的值
    \begin{itemize}
      \item 是否 non-positive
      \item maxReportLatencyUs 不为 0 意味着有 FIFO buffer 存储 sensor events
    \end{itemize}
\end{itemize}

\subsection{Bizarre Usage Examples}
这一节通过实际运行一些反常的例子来进一步了解上述 API 的语义。
\subsubsection{On SensorManager Instance}
\begin{lstlisting}
m1=getSystemService(SENSOR_SERVICE) //get a SensorManager instance
m2=getSystemService(SENSOR_SERVICE) //get a SensorManager instance
//m1==m2
\end{lstlisting}

\subsubsection{On Getting Sensors}
\begin{lstlisting}
m=getSystemService(SENSOR_SERVICE) //get a SensorManager instance
s1=m.getDefaultSensor(SENSOR_TYPE_ACCELEROMETER)  //non-wakeup
s2=m.getDefaultSensor(SENSOR_TYPE_ACCELEROMETER)  //non-wakeup
s3=m.getDefaultSensor(SENSOR_TYPE_ACCELEROMETER,false) //non-wakeup
s4=m.getDefaultSensor(SENSOR_TYPE_ACCELEROMETER,true)  //I have not got this
//s4==null && s1==s2 && s2==s3

s1=m.getDefaultSensor(SENSOR_TYPE_PROXIMITY)  //wakeup
s2=m.getDefaultSensor(SENSOR_TYPE_PROXIMITY)  //wakeup
s3=m.getDefaultSensor(SENSOR_TYPE_PROXIMITY,true) //wakeup
s4=m.getDefaultSensor(SENSOR_TYPE_PROXIMITY,false)  //non-wakeup
//s1==s2 && s2==s3 && s4!=null && s4!=s1

l1=m.getSensorList(SENSOR_TYPE_ALL)
l2=m.getSensorList(SENSOR_TYPE_ALL)
//l1==l2
\end{lstlisting}

\subsubsection{On Registration}
\begin{lstlisting}
SensorEventListener l //suppose initialized
s=m.getDefaultSensor(SENSOR_TYPE_ACCELEROMETER)  //non-wakeup
m.registerListener(l,s,rate)
//debug: m.mSensorListeners.size = 1
m.registerListener(l,s,rate)
//debug: m.mSensorListeners.size = 1
m.unregisterListener(l,s1,rate) //function type of s1 != function type of s
//debug: m.mSensorListeners.size = 1
\end{lstlisting}
