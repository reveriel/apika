
% \title{Sensor API in Android}

\section{Offical Documents for Sensors}

\subsection{SDK referecnce (for developers)}


\href{https://developer.android.com/guide/topics/sensors/sensors_overview.html#sensors-monitor} {Sensor~Overview}
\begin{itemize}
\item
  \href{https://developer.android.com/guide/topics/sensors/sensors_motion.html}
  {Motion Sensors}
  
  The accelerometer and gyroscope sensors are always hardware-based.
  
  The gravity, linear acceleration, rotation vector, significant motion, step
  counter, and step detector sensors are either hardware-based or
  software-based.
   
  
\item
  \href{https://developer.android.com/guide/topics/sensors/sensors_position.html}
  {Position Sensors}

  geomegnetic field sensor, orientation sensor and proximity sensor.

\item
  \href{https://developer.android.com/guide/topics/sensors/sensors_environment.html}
  {Environment Sensors}

  ambient temperature, light, pressure, relative humidity and temperature sensors.
\end{itemize}


Important Classes:

\begin{description}
\item [  \href{https://developer.android.com/reference/android/hardware/SensorManager.html}{SensorManager}]

  You can use this class to create an instance of the sensor service. This class
  provides various methods for accessing and listing sensors, registering and
  unregistering sensor event listeners, and acquiring orientation information.
  This class also provides several sensor constants that are used to report sensor
  accuracy, set data acquisition rates, and calibrate sensors.

\item [
  \href{https://developer.android.com/reference/android/hardware/Sensor.html}{Sensor}]

  You can use this class to create an instance of a specific sensor. This class
  provides various methods that let you determine a sensor's capabilities.

\item[\href{https://developer.android.com/reference/android/hardware/SensorEvent.html}{SensorEvent}]

  The system uses this class to create a sensor event object, which provides
  information about a sensor event. A sensor event object includes the following
  information: the raw sensor data, the type of sensor that generated the event,
  the accuracy of the data, and the timestamp for the event.

\item[\href{https://developer.android.com/reference/android/hardware/SensorEventListener.html}{SensorEventListener}]

  You can use this interface to create two callback methods that receive
  notifications (sensor events) when sensor values change or when sensor accuracy
  changes.
\end{description}


{
  \small
  \paragraph{note:}
  \begin{itemize}

  \item \textbf{Unregister sensor listeners} from document:
    \href{https://developer.android.com/guide/topics/sensors/sensors_overview.html#sensors-monitor}
    {Sensors Overview}

  Always make sure to disable sensors you don't need especially when your
  activity is paused. Failling to do so can drain the battery in just a few hours.
  Note that the system will not desable sensors automatically when the screen
  turns off.

  It's also important to note that this example uses the onResume() and onPause()
  callback methods to register and unregister the sensor event listener. As a best
  practice you should always disable sensors you don't need, especially when your
  activity is paused. Failing to do so can drain the battery in just a few hours
  because some sensors have substantial power requirements and can use up battery
  power quickly. The system will not disable sensors automatically when the screen
  turns off.
  
\item 
\textbf{Don't block the onSensorChanged() method}
  from document: \href{https://developer.android.com/guide/topics/sensors/sensors_overview.html#sensors-practices}
  {Sensors Overview}

  Sensor data can change at a high rate, which means the system may call the
  onSensorChanged(SensorEvent) method quite often. As a best practice, you should
  do as little as possible within the onSensorChanged(SensorEvent) method so you
  don't block it. If your application requires you to do any data filtering or
  reduction of sensor data, you should perform that work outside of the
  onSensorChanged(SensorEvent) method.

\item 
  \textbf{Wakelock for non-wake-up sensors}
  from document: \href{https://developer.android.com/reference/android/hardware/SensorManager.html#registerListener(android.hardware.SensorEventListener,%20android.hardware.Sensor,%20int,%20int)}
    {SensorManager}


  In the case of non-wake-up sensors, the events are only delivered while the
  Application Processor (AP) is not in suspend mode.
  %See isWakeUpSensor() for more details.
  To ensure delivery of events from non-wake-up sensors even when the
  screen is OFF, the application registering to the sensor must hold a partial
  wake-lock to keep the AP awAke, otherwise some events might be lost while the AP
  is asleep. Note that although events might be lost while the AP is asleep, the
  sensor will still consume power if it is not explicitly deactivated by the
  application.

  \end{itemize}
}

\subsection{Documents targeted at manufacturers}

\href{https://source.android.com/devices/sensors/index.html}{Sensors}
\begin{itemize}
\item \href{https://source.android.com/devices/sensors/suspend-mode.html}
  {Suspend mode} (non-wake-up and wake-up sensors)
  
\item \href{https://source.android.com/devices/sensors/batching.html}
  {Batching} (FIFO)
\end{itemize}


\subsection{Sensor's type}

\textbf{Type by functionality.} 
\vspace*{.5cm}

{\scriptsize
\begin{tabular}{llll}
% TYPE_DEVICE_PRIVATE_BASE
% The lowest sensor type vendor defined sensors can use.
  \toprule
  Constant                            & sensor type                                 & API level & others                      \\
  \midrule
  TYPE\_ACCELEROMETER                 & an accelerometer sensor type                & 3         &                           \\
  TYPE\_ACCELEROMETER\_UNCALIBRATED   & an uncalibrated accelerometer sensor        & O         &                           \\
  TYPE\_ALL                           & all sensor types                            & 3         &                           \\
  TYPE\_AMBIENT\_TEMPERATURE          & an ambient temperature sensor type          & 14        &                           \\
  TYPE\_GAME\_ROTATION\_VECTOR        & an uncalibrated rotation vector sensor type & 18        &                           \\
  TYPE\_GEOMAGNETIC\_ROTATION\_VECTOR & a geo-magnetic rotation vector              & 19        &                           \\
  TYPE\_GRAVITY                       & a gravity sensor type                       & 9         &                           \\
  TYPE\_GYROSCOPE                     & a gyroscope sensor type                     & 3         &                           \\
  TYPE\_GYROSCOPE\_UNCALIBRATED       & an uncalibrated gyroscope sensor type       & 18        &                           \\
  TYPE\_HEART\_BEAT                   & a motion detect sensor                      & 24        &                           \\
  TYPE\_HEART\_RATE                   & a heart rate monitor                        & 20        &                           \\
  TYPE\_LIGHT                         & a light sensor type                         & 3         &                           \\
  TYPE\_LINEAR\_ACCELERATION          & a linear acceleration sensor type           & 9         &                           \\
  TYPE\_LOW\_LATENCY\_OFFBODY\_DETECT & a low latency off-body detect sensor        & O         &                           \\
  TYPE\_MAGNETIC\_FIELD               & a magnetic field sensor type                & 3         &                           \\
  TYPE\_MAGNETIC\_FIELD\_UNCALIBRATED & an uncalibrated magnetic field sensor type  & 18        &                           \\
  TYPE\_MOTION\_DETECT                & a motion detect sensor                      & 24        &                           \\
  TYPE\_ORIENTATION                   &                                             & 3         & Deprecated in API level 8 \\
  TYPE\_POSE\_6DOF                    & a pose sensor with 6 degrees of freedom     & 24        &                           \\
  TYPE\_PRESSURE                      & a pressure sensor type                      & 3         &                           \\
  TYPE\_PROXIMITY                     & a proximity sensor type                     & 3         & wake up sensor            \\
  TYPE\_RELATIVE\_HUMIDITY            & a relative humidity sensor type             & 14        &                           \\
  TYPE\_ROTATION\_VECTOR              & a rotation vector sensor type               & 9         &                           \\
  TYPE\_SIGNIFICANT\_MOTION           & a significant motion trigger sensor         & 18        & wake up sensor            \\
  TYPE\_STATIONARY\_DETECT            & a stationary detect sensor                  & 24        &                           \\
  TYPE\_STEP\_COUNTER                 & a step counter sensor                       & 19        &                           \\
  TYPE\_STEP\_DETECTOR                & a step detector sensor                      & 19        &                           \\
  TYPE\_TEMPERATURE                   &                                             & 3         & Deprecated in API 14      \\
  \bottomrule
\end{tabular} 
}

\textbf{Type by hardware feature}
\begin{itemize}
\item \textbf{Non-wake-up sensors}

  Non-wake-up sensors are sensors that do not prevent the SoC from going into
  suspend mode and do not wake the SoC up to report data. In particular, the
  drivers are not allowed to hold wake-locks. It is the responsibility of
  applications to keep a partial wake lock should they wish to receive events
  from non-wake-up sensors while the screen is off. While the SoC is in suspend
  mode, the sensors must continue to function and generate events, which are put
  in a hardware FIFO.
  %(See Batching for more details.)
  The events in the FIFO are delivered to the applications when the SoC wakes
  up. If the FIFO is too small to store all events, the older events are lost;
  the oldest data is dropped to accommodate the latest data. In the extreme case
  where the FIFO is nonexistent, all events generated while the SoC is in
  suspend mode are lost. One exception is the latest event from each on-change
  sensor: the last event must be saved outside of the FIFO so it cannot be lost.

  As soon as the SoC gets out of suspend mode, all events from the FIFO are
  reported and operations resume as normal.

  Applications using non-wake-up sensors should either hold a wake lock to
  ensure the system doesn't go to suspend, unregister from the sensors when they
  do not need them, or expect to lose events while the SoC is in suspend mode.

\item \textbf{Wake-up sensors}

  In opposition to non-wake-up sensors, wake-up sensors ensure that their data
  is delivered independently of the state of the SoC. While the SoC is awake,
  the wake-up sensors behave like non-wake-up-sensors. When the SoC is asleep,
  wake-up sensors must wake up the SoC to deliver events. They must still let
  the SoC go into suspend mode, but must also wake it up when an event needs to
  be reported. That is, the sensor must wake the SoC up and deliver the events
  before the maximum reporting latency has elapsed or the hardware FIFO gets
  full. %See Batching for more details.

  To ensure the applications have the time to receive the event before the SoC
  goes back to sleep, the driver must hold a "timeout wake lock" for 200
  milliseconds each time an event is being reported. That is, the SoC should not
  be allowed to go back to sleep in the 200 milliseconds following a wake-up
  interrupt. This requirement will disappear in a future Android release, and we
  need this timeout wake lock until then.
\end{itemize}

\subsection{How to get a wake-up or a non-wake-up sensor}

\subimport{./}{SensorManagerAPI}



