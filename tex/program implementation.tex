\usepackage{color}
\documentclass{article}
\title{On Statistics Program}
\begin{document}
\tableofcontents
\section{Program Input}
The command line is
\begin{verbatime}
java apika.Main <APKName>
\end{verbatime}
The program also reads a file named 
\textcolor{blue}{config.properties} located 
located at the current working directory. 
\par{
The contents of the \textcolor{blue}{config.properties} 
file is of the form.
\begin{verbatime}
<PropertityName> = <PropertityValues>
......
\end{verbatime}
For the time being the \textcolor{blue}{PropertityName} can be
\begin{itemize}
	\item ANDROID_JARS : the path of android sdk(realized)
\end{itemize}
}

\section{Kinds of Statistics}
\subsubsection{Component Usage}
Component base class list is
\begin{itemize}
	\item Activity
		\begin{itemize}
			\item AccountAuthenticatorActivity
			\item ActivityGroup
				\begin{itemize}
					\item TabActivity
				\end{itemize}
			\item AliasActivity
			\item ExpandableListActivity
			\item FragmentActivity 
				\begin{itemize}
					\item AppCompatActivity
				\end{itemize}
			\item ListActivity
				\begin{itemize}
					\item LauncherActivity
					\item PreferenceActivity
				\end{itemize}
			\item NativeActivity
		\end{itemize}
	\item Service
		\begin{itemize}
			\item AbstractInputMethodService
				\begin{itemize}
					\item InputMethodService
				\end{itemize}
			\item AccessibilityService
			\item AutoFillService
			\item CallScreeningService
			\item CameraPrewarmService
			\item CarrierMessagingService
			\item CarrierService
			\item ChooserTargetService
			\item ConditionProviderService
			\item ConnectionService
			\item CustomTabService
			\item DreamService
			\item HostApduService
			\item HostNfcFService
			\item InCallService
			\item IntentService
			\item JobService
			\item MediaBrowserService
			\item MediaBrowserServiceCompat
			\item MediaRouteProviderService
			\item MidiDeviceService
			\item NotificationCompatSideChannelService
			\item NotificationListenerService
			\item OffHostApduService
			\item PostMessageService
			\item PrintService
			\item RecognitionService
			\item RemoteViewsService
			\item SettingInjectorService
			\item SpellCheckerService
			\item TextToSpeechService
			\item TileService
			\item TvInputService
			\item VisualVoicemailService
			\item VoiceInteractionService
			\item VoiceInteractionSessionService
			\item VpnService
			\item VrListenerService
			\item WallpaperService
		\end{itemize}
	\item BroadcastReceiver
		\begin{itemize}
			\item AppWidgetProvider
			\item DeviceAdminReceiver
			\item MediaButtonReceiver
			\item RestrictionsReceiver
			\item WakefulBroadcastReceiver
		\end{itemize}
	\item ContentProvider
		\begin{itemize}
			\item DocumentsProvider
			\item FileProvider
			\item MockContentProvider
			\item SearchRecentSuggestionsProvider
		\end{itemize}
\end{itemize}
(There're so many (virtual) methods. Maybe we can 
get the virtual methods list with soot.)
\par{
This kind of statistics can be generalized to 
statistics on class hierachy.
Properties like
\begin{verbatime}
HierarchyBaseClasses = c1;c2;c3...
HierarchyCollectVirtualMethodOverride = true/false
HierarchyCollect* = *
\end{verbatime}
can be added to the \textcolor{blue}{config.properties} file, 
the program reads the list, find all virtual methods defined,
and get the statistics according to the related options.
}
\subsubsection{SensorEventListener Usage}
(These are interfaces)Class list is 
\begin{itemize}
	\item SensorEventListener
		\begin{itemize}
			\item SensorEventListener2
				\begin{itemize}
					\item SensorEventCallback
				\end{itemize}
		\end{itemize}
\end{itemize}
This kind of statistics covers the hierarchy so 
we can run the previous statistics on these 
classes/interfaces,
but we also want more detailed information like,
\begin{itemize}
	\item number of invocation point
	 	\begin{itemize}
	 		\item args are constant?
	 		\item caller
	 	\end{itemize}
\end{itemize}
So a property like
\begin{verbatime}
HierarchyBaseClassesDetailed = c1;c2;c3...
\end{verbatime}

\subsubsection{Sensor and Wakelock Usage}
The previous statistics applies here, as when 
try to find the subclasses of Sensor the result 
may be that there're no subclasses. 
We can use a line like
\begin{verbatime}
<ClassName> = method1;method2;method3...
\end{verbatime}
to indicate the (public) methods we are interested 
in. And we can also find all the (public) methods 
with soot when the amount is too large.
\par
{
	Here, WakeLock related methods are
	\begin{itemize}
		\item void acquire()
		\item void acquire(long timeout)
		\item void release()
		\item void release(int flags)
	\end{itemize}
}
\par
{
	\begin{itemize}
		\item boolean isWakeUpSensor()
	\end{itemize}
}

\subsubsection{SensorManager and PowerManager}
Interesting method list of PowerManager is
\begin{itemize}
	\item WakeLock newWakeLock(int levelAndFlags, String tag)
\end{itemize}
%\begin{itemize}
%	\item Sensor getDefaultSensor(int type)
%	\item Sensor getDefaultSensor(int type)
%	\item List<Sensor> getSensorList(int type)
%	\item int getSensors() (deprecated)
%	\item boolean registerListener(SensorEventListener listener, Sensor sensor, int samplingPeriodUs)
%	\item boolean registerListener(SensorEventListener listener, Sensor sensor, int samplingPeriodUs, int maxReportLatencyUs)
%	\item boolean registerListener(SensorEventListener listener, Sensor sensor, int samplingPeriodUs, Handler handler)
%	\item boolean registerListener(SensorEventListener listener, Sensor sensor, int samplingPeriodUs, int maxReportLatencyUs, Handler handler)
%\end{itemize}