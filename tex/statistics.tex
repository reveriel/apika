\documentclass{article}
\title{On Statistics}
\begin{document}
\tableofcontents
\section{Overview}
So far, we want to get the statistics of
\begin{itemize}
	\item deprecated methods usage
	\item number of component and their names
	\item wake lock usage
	\item sensor type used
\end{itemize}
\par
{
I find it difficult to get a knowledge of dexlib ... 
After all, I've already been learning soot for hours and can manage 
to visit each statement in a java class file. 
So I decided to use soot to get the statistics. 
The procedure goes like:
\begin{itemize}
	\item decompile apk and transform them into jar with dex2jar
	\item get the statistics with soot or by checking the manifest.xml file
\end{itemize}
}

\section{Statistics of Method Usage}
This can be used for the statistics of 
\begin{itemize}
	\item sensor type usage
	\item wake lock usage
	\item deprecated method usage
\end{itemize}
\par{
To get a statistics of method usage, we need to specify the 
\begin{itemize}
	\item signature of the method
		\begin{itemize}
		\item in Jimple, the method signature is of the form
			\begin{verbatim}<PackageName.ClassName: Type MethodName(pars)>\end{verbatim}
			, maybe there can be a program that dump all the statements of a java class file, 
			and one can write an android program containing the methods interested and convert 
			the apk to class to feed the dump-all-statement program to see what's the signature
		\end{itemize}
	\item Java class names to visit and the jar path containing the classes
		\begin{itemize}
			\item There're lots of classes for a statistics, it's better to specify a 
			root path and the program recursively find all the jars and the classes
		\end{itemize}
	\item environment for soot, in detail, path for jars including but not limitted to
		\begin{itemize}
			\item rt.jar (from jdk/jre)
			\item jsse.jar (from jdk/jre Linux only)
			\item jce.jar (from jdk/jre)
			\item android.jar (from android sdk)
		\end{itemize}
\end{itemize}
}
\section{Statistics of Component Usage}
It seemed that we can get this statistics from the manifest.xml file, 
that is there's the claim that one cannot create a component with android code. 
If so, it would be easy to get this kind of statistics. 
But first need to make sure whether the claim is correct or not.

\end{document}
